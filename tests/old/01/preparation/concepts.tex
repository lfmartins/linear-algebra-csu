\documentclass[12pt]{article}

\input{../../../../../fimacros.tex}

\setheadings{MTH 288 --- Linear Algebra --- Test 1 Concepts}

\begin{document}

\section{Questions}
For each of the following statements determine if it is true or false, and provide a brief justification for your answer.

It is \emph{guaranteed} that one or more of these questions will be in the test, in \emph{exactly the same form} as they appear below.


\begin{enumerate}

\item A linear system with 3 equations and 2 unknowns cannot have solutions.

\item A linear system with 3 equations and 5 unknowns always has solutions.

\item Every matrix can be transformed into an unique equivalent matrix in row echelon form.

\item Every matrix can be transformed into an unique equivalent matrix in reduced row echelon form.


\item A homogeneous linear system always has at least one solution.

\item It is possible for a linear system to have exactly 10 solutions.

\item If a homogeneous system has 5 equations and 8 unknowns, then it must have infinitely many solutions.

\item Suppose that a vector $\mathbf{v}$ satisfies the following identity:
\[
\mathbf{v} = 2(\mathbf{u}_1-\mathbf{u}_2) + 3(\mathbf{u}_2-\mathbf{u}_3) - 4(\mathbf{u}_3-\mathbf{u}_1)
\]
Then, $\mathbf{v}$ is in the span of $\{\mathbf{u}_1,\mathbf{u}_2,\mathbf{u}_3\}$.


\item Let $\mathbf{u}_1$, $\mathbf{u}_2$ and $\mathbf{u}_3$ be three arbitrary vectors. Then $\mathbf{v}=\mathbf{u}_1+\mathbf{u}_2$ is in the span of $\{\mathbf{u}_1, \mathbf{u}_2, \mathbf{u}_3\}$. 

\item A set of 4 vectors, $\{\mathbf{u}_1, \mathbf{u}_2, \mathbf{u}_3, \mathbf{u}_4\}$, can span the whole euclidean space $\R^7$.

\item The vector $\mathbf{u}_1$ is always in the span of $\{\mathbf{u}_1, \mathbf{u}_2\}$

\item Let $\{\mathbf{u}_1, \mathbf{u}_2, \mathbf{u}_3,\ldots, \mathbf{u}_8\}$ be a set of vectors in $\R^5$, and let 
$\mathbf{A}=[\begin{matrix} \mathbf{u}_1 & \mathbf{u}_2 & \mathbf{u}_3 & \cdots & \mathbf{u}_8\end{matrix}]$ be the matrix that has the given vectors as columns. 
Suppose that, given an arbitrary vector $\mathbf{b}$, the augmented matrix of the system $\mathbf{A}\mathbf{x}=\mathbf{b}$ is equivalent to the following reduced row echelon matrix: 
\[
\left[\begin{matrix}
1&0&0&2&0&0&0&-1 & b_1'\\ 
0&0&1&3&0&0&0&2 & b_2'\\ 
0&0&0&0&1&0&0&0& b_3'\\ 
0&0&0&0&0&1&0&4& b_4'\\ 
0&0&0&0&0&0&1&-3& b_5'\\ 
\end{matrix}\right]
\]
where $b_1',b_2',\ldots,b_5'$ are real numbers. Then, the vectors $\{\mathbf{u}_1, \mathbf{u}_2, \mathbf{u}_3,\ldots, \mathbf{u}_8\}$ span $\R^5$.

\item Let $\{\mathbf{u}_1, \mathbf{u}_2, \mathbf{u}_3, \mathbf{u}_4\}$ be a set of vectors in $\R^4$, and let 
$\mathbf{A}=[\begin{matrix} \mathbf{u}_1 & \mathbf{u}_2 & \mathbf{u}_3 & \mathbf{u}_4\end{matrix}]$ be the matrix that has the given vectors as columns. Given an arbitrary vector $\mathbf{b}$, the augmented matrix of the system $\mathbf{A}\mathbf{x}=\mathbf{b}$ is equivalent to the following reduced row echelon matrix: 
\[
\left[\begin{matrix}
1&0&0&4 & b_1'\\ 
0&1&0&-2& b_2'\\ 
0&0&1& 2& b_3'\\ 
0&0&0&0& b_4'\\ 
\end{matrix}\right]
\]
where $b_1',b_2',b_3',b_4'$ are real numbers. Then, the vectors $\{\mathbf{u}_1, \mathbf{u}_2, \mathbf{u}_3,\mathbf{u}_4\}$ span $\R^4$.

\item To span the whole euclidean space $\R^9$, we need at least 10 vectors.

\item The zero vector, $\mathbf{0}$, is always in the span of the set of vectors $\{\mathbf{u}_1, \mathbf{u}_2, \mathbf{u}_3,\ldots ,\mathbf{u}_m\}$

\item The vectors $\{\mathbf{u}_1, \mathbf{u}_2, \mathbf{u}_3,\ldots ,\mathbf{u}_m\}$ are linearly independent if and only if there are scalars $x_1, x_2,\ldots x_m$ such that:
\[
x_1\mathbf{u}_1+x_2\mathbf{u}_2+\cdots+x_m\mathbf{u}_m=\mathbf{0}
\]

\item The vectors $\mathbf{u}$, $\mathbf{v}$ and $\mathbf{u}+\mathbf{v}$ are always linearly dependent.

\item A set of $6$ vectors in the euclidean space $\R^4$ is always linearly dependent.

\item A set of $3$ vectors in the euclidean space $\R^4$ is always linearly independent. 

\item Suppose that the set of vectors $\{\mathbf{u}_1, \mathbf{u}_2, \mathbf{u}_3,\ldots ,\mathbf{u}_m\}$ spans the euclidean space $\R^n$. Then, this set of vectors must be linearly independent.

\item If $\{\mathbf{u}_1, \mathbf{u}_2, \mathbf{u}_3\}$ is linearly dependent, then so is $\{\mathbf{u}_1, \mathbf{u}_2, \mathbf{u}_3, \mathbf{u}_4\}$.

\item If $\{\mathbf{u}_1, \mathbf{u}_2, \mathbf{u}_3\}$ is linearly independent, then so is $\{\mathbf{u}_1, \mathbf{u}_2, \mathbf{u}_3, \mathbf{u}_4\}$.

\item If $\{\mathbf{u}_1, \mathbf{u}_2, \mathbf{u}_3, \mathbf{u}_4\}$
is linearly independent, then so is $\{\mathbf{u}_1, \mathbf{u}_2, \mathbf{u}_3\}$.

\item The set of vectors $\{\mathbf{u}_1, \mathbf{u}_2, 2\mathbf{u}_1\}$ is linearly independent.

\item The set of vectors $\{\mathbf{0}, \mathbf{u}_2, \mathbf{u}_3\}$ is linearly dependent.

\item Let $\{\mathbf{u}_1, \mathbf{u}_2, \mathbf{u}_3\}$ be a set of vectors in $\R^5$, and let
$\mathbf{A}=[\begin{matrix} \mathbf{u}_1&\mathbf{u}_2&\mathbf{u}_3\end{matrix}]$ be the matrix that has these vectors as columns. 
Suppose that the augmented matrix of the system $\mathbf{A}\mathbf{x}=\mathbf{0}$ is equivalent to the following reduced row echelon matrix:
\[
\left[\begin{matrix}
1&0&0&0\\ 
0&1&0&0\\ 
0&0&1&0\\ 
0&0&0&0\\ 
0&0&0&0\\ 
\end{matrix}\right]
\]
Then, the vectors $\{\mathbf{u}_1, \mathbf{u}_2, \mathbf{u}_3\}$ are linearly independent.

\item Let $\{\mathbf{u}_1, \mathbf{u}_2, \mathbf{u}_3\}$ be a set of vectors in $\R^5$, and let
$\mathbf{A}=[\begin{matrix} \mathbf{u}_1&\mathbf{u}_2&\mathbf{u}_3\end{matrix}]$ be the matrix that has these vectors as columns. 
Suppose that the augmented matrix of the system $\mathbf{A}\mathbf{x}=\mathbf{0}$ is equivalent to the following reduced row echelon matrix:
\[
\left[\begin{matrix}
1&0&0&0\\ 
0&0&0&0\\ 
0&0&1&0\\ 
0&0&0&0\\ 
0&0&0&0\\ 
\end{matrix}\right]
\]
Then, the vectors $\{\mathbf{u}_1, \mathbf{u}_2, \mathbf{u}_3\}$ are linearly independent.

\item Suppose that the homogeneous system $\mathbf{A}\mathbf{x}=\mathbf{0}$ only has the trivial solution. Then, for any vector $\mathbf{b}$, the system $\mathbf{A}\mathbf{x}=\mathbf{b}$ has exactly one solution.

\item Suppose that the homogeneous system $\mathbf{A}\mathbf{x}=\mathbf{0}$ has infinitely many solutions. Then, for any vector $\mathbf{b}$, the system $\mathbf{A}\mathbf{x}=\mathbf{b}$ always has infinitely many solutions.

\item Suppose that the homogeneous system $\mathbf{A}\mathbf{x}=\mathbf{0}$ only has the trivial solution. Then, there may exist vectors $\mathbf{b}$, such that the system $\mathbf{A}\mathbf{x}=\mathbf{b}$ does not have any solution.

\end{enumerate}

\clearpage

\section{Answers}

\begin{enumerate}

\item A linear system with 3 equations and 2 unknowns cannot have solutions.\\
\textbf{False}. Consider the system:
\[
\begin{matrix}
x_1=1\\
x_2=2\\
x_1+x_2=3
\end{matrix}
\]
If the first two equations are true, than the third one is also obviously true. So this system has the solution $x_1=1$, $x_2=2$.

\item A linear system with 3 equations and 5 unknowns always has solutions.\\
\textbf{False}. Consider the (rather silly) linear system:
\[
\begin{matrix}
x_1+x_2+x_3+x_4+x_5=1\\
x_1+x_2+x_3+x_4+x_5=2\\
x_1+x_2+x_3+x_4+x_5=3\\
\end{matrix}
\]
Since these equalities are obviously contradictory, the system has no solutions.

\item Every matrix can be transformed into an unique equivalent matrix in row echelon form.
\textbf{False}. Different sequences of row operations can lead to different row echelon forms.

\item Every matrix can be transformed into an unique equivalent matrix in reduced row echelon form.\\
\textbf{True}. This is Theorem 1.6 (page 23) of the textbook.

\item A homogeneous linear system always has at least one solution.\\
\textbf{True}. A $m\times n$ homogeneous linear system always has the trivial solution $x_1=x_2=\cdots=x_n=0$

\item It is possible for a linear system to have exactly 10 solutions.\\
\textbf{False}. A linear system can have zero, one or infinitely many solutions.

\item If a homogeneous system has 5 equations and 8 unknowns, then it must have infinitely many solutions.\\
\textbf{True}. A homogeneous system that has more unknowns than equations always has infinitely many solutions.  

\item Suppose that a vector $\mathbf{v}$ satisfies the following identity:
\[
\mathbf{v} = 2(\mathbf{u}_1-\mathbf{u}_2) + 3(\mathbf{u}_2-\mathbf{u}_3) - 4(\mathbf{u}_3-\mathbf{u}_1)
\]
Then, $\mathbf{v}$ is in the span of $\{\mathbf{u}_1,\mathbf{u}_2,\mathbf{u}_3\}$.\\
\textbf{True}. Using the rules for vector operations, the equality above can be written as:
\[
\mathbf{v}=6\mathbf{u}_1+\mathbf{u}_2-7\mathbf{u}_3,
\]
so that $\mathbf{v}$ is a linear combination of $\mathbf{u}_1,\mathbf{u}_2,\mathbf{u}_3$.


\item Let $\mathbf{u}_1$, $\mathbf{u}_2$ and $\mathbf{u}_3$ be three arbitrary vectors. Then $\mathbf{v}=\mathbf{u}_1+\mathbf{u}_2$ is in the span of $\{\mathbf{u}_1, \mathbf{u}_2, \mathbf{u}_3\}$.\\
\textbf{True}. We can write $\mathbf{v}=\mathbf{u}_1+\mathbf{u}_2+0\mathbf{u_3}$, so $\mathbf{v}$ is a linear combination of $\mathbf{u}_1, \mathbf{u}_2, \mathbf{u}_3$.

\item A set of 4 vectors, $\{\mathbf{u}_1, \mathbf{u}_2, \mathbf{u}_3, \mathbf{u}_4\}$ can span the whole euclidean space $\R^7$.\\
\textbf{False}. To span $\R^7$, we need at least 7 vectors.

\item The vector $\mathbf{u}_1$ is always in the span of $\{\mathbf{u}_1, \mathbf{u}_2\}$\\
\textbf{True}. We can write $\mathbf{u}_1$ as the linear combination $1\mathbf{u}_1+0\mathbf{u}_2$.

\item Let $\{\mathbf{u}_1, \mathbf{u}_2, \mathbf{u}_3,\ldots, \mathbf{u}_8\}$ be a set of vectors in $\R^5$, and let 
$\mathbf{A}=[\begin{matrix} \mathbf{u}_1 & \mathbf{u}_2 & \mathbf{u}_3 & \cdots & \mathbf{u}_8\end{matrix}]$ be the matrix that has the given vectors as columns. 
Suppose that, given an arbitrary vector $\mathbf{b}$, the augmented matrix of the system $\mathbf{A}\mathbf{x}=\mathbf{b}$ is equivalent to the following reduced row echelon matrix: 
\[
\left[\begin{matrix}
1&0&0&2&0&0&0&-1 & b_1'\\ 
0&0&1&3&0&0&0&2 & b_2'\\ 
0&0&0&0&1&0&0&0& b_3'\\ 
0&0&0&0&0&1&0&4& b_4'\\ 
0&0&0&0&0&0&1&-3& b_5'\\ 
\end{matrix}\right]
\]
where $b_1',b_2',\ldots,b_5'$ are real numbers. Then, the vectors $\{\mathbf{u}_1, \mathbf{u}_2, \mathbf{u}_3,\ldots, \mathbf{u}_8\}$ span $\R^5$.\\
\textbf{True}. The matrix above corresponds to the system:
\[
\begin{matrix}
x_1+2x_4-x_8=b_1'\\
x_3+3x_4+2x_8=b_2'\\
x_5=b_3'\\
x_6+4x_8=b_4'\\
x_7-3x_8=b_5'
\end{matrix}
\]
In this system, the variables $x_2$, $x_4$ and $x_8$ are free, and can be chosen arbitrarily. Once $x_2$, $x_4$ and $x_8$ are chosen, the other variables are uniquely determined, so that the system always has solutions.

\item Let $\{\mathbf{u}_1, \mathbf{u}_2, \mathbf{u}_3, \mathbf{u}_4\}$ be a set of vectors in $\R^5$, and let 
$\mathbf{A}=[\begin{matrix} \mathbf{u}_1 & \mathbf{u}_2 & \mathbf{u}_3 & \mathbf{u}_4 \end{matrix}]$ be the matrix that has the given vectors as columns. Given an arbitrary vector $\mathbf{b}$, the augmented matrix of the system $\mathbf{A}\mathbf{x}=\mathbf{b}$ is equivalent to the following reduced row echelon matrix: 
\[
\left[\begin{matrix}
1&0&0&4 & b_1'\\ 
0&1&0&-2& b_2'\\ 
0&0&1& 2& b_3'\\ 
0&0&0&0& b_4'\\ 
\end{matrix}\right]
\]
where $b_1',b_2',b_3',b_4'$ are real numbers. Then, the vectors $\{\mathbf{u}_1, \mathbf{u}_2, \mathbf{u}_3,\mathbf{u}_4\}$ span $\R^5$.\\
\textbf{False}. The last equation of the system corresponding to the matrix above is $0=b_4'$, which cannot be satisfied if $b_4'$ is not zero.


\item To span the whole euclidean space $\R^9$, we need at least 10 vectors.\\
\textbf{False}. 9 vectors are sufficient to span $\R^9$.

\item The zero vector, $\mathbf{0}$, is always in the span of any set of vectors $\{\mathbf{u}_1, \mathbf{u}_2, \mathbf{u}_3,\ldots ,\mathbf{u}_m\}$\\
\textbf{True}. We can write the zero vector as a linear combination of any set of vectors:
\[
0\mathbf{u}_1+0\mathbf{u}_2+0\mathbf{u}_3+\cdots+0\mathbf{u}_m=\mathbf{0}
\]

\item The vectors $\{\mathbf{u}_1, \mathbf{u}_2, \mathbf{u}_3,\ldots ,\mathbf{u}_m\}$ are linearly independent if and only if there are scalars $x_1, x_2,\ldots x_m$ such that:
\[
x_1\mathbf{u}_1+x_2\mathbf{u}_2+\cdots+x_m\mathbf{u}_m=\mathbf{0}
\]\\
\textbf{False}. We can always write the zero vector as a linear combination of any set of vectors. The vectors are linearly independent if the only possible such linear combination is the trivial one, in which all scalars are zero.

\item The vectors $\mathbf{u}$, $\mathbf{v}$ and $\mathbf{u}+\mathbf{v}$ are always linearly dependent.\\
\textbf{True}. The following gives a nontrivial linear combination of these three vectors that is equal to the zero vector:
\[
1\mathbf{u}+1\mathbf{v}+(-1)(\mathbf{u}+\mathbf{v})=\mathbf{0}
\]

\item A set of $6$ vectors in the euclidean space $\R^4$ is always linearly dependent.\\
\textbf{True}. If $m>n$, then any set of $m$ vectors in $\R^n$ is linearly dependent.

\item A set of $3$ vectors in the euclidean space $\R^4$ is always linearly independent.\\
\textbf{False}. Consider the vectors:
\[
\left[\begin{matrix} 1 \\ 0 \\ 0 \\ 0 \end{matrix}\right]\quad
\left[\begin{matrix} 0 \\ 1 \\ 0 \\ 0 \end{matrix}\right]\quad
\left[\begin{matrix} 1 \\ 1 \\ 0 \\ 0 \end{matrix}\right]
\]
Since the third vector is the sum of the first two, they are not linearly independent.

\item Suppose that the set of vectors $\{\mathbf{u}_1, \mathbf{u}_2, \mathbf{u}_3,\ldots ,\mathbf{u}_m\}$ spans the euclidean space $\R^n$. Then, this set of vectors must be linearly independent.\\
\textbf{False}. The following set of four vectors spans $\R^3$:
\[
\left[\begin{matrix} 1 \\ 0 \\ 0 \end{matrix}\right]\quad
\left[\begin{matrix} 0 \\ 1 \\ 0 \end{matrix}\right]\quad
\left[\begin{matrix} 0 \\ 0 \\ 1 \end{matrix}\right]\quad
\left[\begin{matrix} 1 \\ 1 \\ 1 \end{matrix}\right]\quad
\]
(In fact, it is enough to use the first three vectors to span $\R^3$.) However, the fourth vector is a linear combination of the first three, so the set is linearly dependent.

\item If $\{\mathbf{u}_1, \mathbf{u}_2, \mathbf{u}_3\}$ is linearly dependent, then so is $\{\mathbf{u}_1, \mathbf{u}_2, \mathbf{u}_3, \mathbf{u}_4\}$.\\
\textbf{True}. Suppose that we have a nontrivial linear combination of the three first vectors that is equal to the zero vector:
\[
x_1\mathbf{u}_1+x_2\mathbf{u}_2+x_3\mathbf{u}_3=\mathbf{0},\quad\text{where $x_1, x_2, x_3$ are not all zero}.
\]
Then, the following is a nontrivial linear combination of the four vectors that is equal to zero:
\[
x_1\mathbf{u}_1+x_2\mathbf{u}_2+x_3\mathbf{u}_3+0\mathbf{u}_4=\mathbf{0}.
\]
Notice that the fact that the coefficient of $\mathbf{u}_4$ is zero does not matter, because one of the scalars $x_1,x_2,x_3$ must be nonzero.

\item If $\{\mathbf{u}_1, \mathbf{u}_2, \mathbf{u}_3\}$ is linearly independent, then so is $\{\mathbf{u}_1, \mathbf{u}_2, \mathbf{u}_3, \mathbf{u}_4\}$.
\textbf{False}. If we let $\mathbf{u}_4$ be any linear combination of the other three vectors, then the second set is linearly dependent.

\item If $\{\mathbf{u}_1, \mathbf{u}_2, \mathbf{u}_3, \mathbf{u}_4\}$
is linearly independent, then so is $\{\mathbf{u}_1, \mathbf{u}_2, \mathbf{u}_3\}$.\\
\textbf{True}. Suppose that we have a linear combination of $\{\mathbf{u}_1, \mathbf{u}_2, \mathbf{u}_3\}$ that is equal to the zero vector:
\[
x_1\mathbf{u}_1+x_2\mathbf{u}_2+x_3\mathbf{u}_3=\mathbf{0}
\]
Then, the following is also true:
\[
x_1\mathbf{u}_1+x_2\mathbf{u}_2+x_3\mathbf{u}_3+0\mathbf{u}_4=\mathbf{0}.
\]
Since the set of four vectors $\{\mathbf{u}_1, \mathbf{u}_2, \mathbf{u}_3, \mathbf{u}_4\}$ is linearly independent, all coefficients in the above equation must be zero. But this shows that only a trivial linear combination of the set of three vectors $\{\mathbf{u}_1, \mathbf{u}_2, \mathbf{u}_3\}$ can be equal to the zero vector.

\item The set of vectors $\{\mathbf{u}_1, \mathbf{u}_2, 2\mathbf{u}_1\}$ is linearly independent.\\
\textbf{False}. The third vector is a linear combination of the first two:
\[
2\mathbf{u}_1=2\mathbf{u}_1+0\mathbf{u_2}
\]
Thus, the vectors are linearly dependent.

\item The set of vectors $\{\mathbf{0}, \mathbf{u}_2, \mathbf{u}_3\}$ is linearly dependent.\\
\textbf{True}. The following nontrivial linear combination of the vectors is equal to the zero vector:
\[
1\cdot\mathbf{0}+0\cdot\mathbf{u}_3+0\cdot\mathbf{u}_3=\mathbf{0}
\]

\item Let $\{\mathbf{u}_1, \mathbf{u}_2, \mathbf{u}_3\}$ be a set of vectors in $\R^5$, and let
$\mathbf{A}=[\begin{matrix} \mathbf{u}_1&\mathbf{u}_2&\mathbf{u}_3\end{matrix}]$ be the matrix that has these vectors as columns. 
Suppose that the augmented matrix of the system $\mathbf{A}\mathbf{x}=\mathbf{0}$ is equivalent to the following reduced row echelon matrix:
\[
\left[\begin{matrix}
1&0&0&0\\ 
0&1&0&0\\ 
0&0&1&0\\ 
0&0&0&0\\ 
0&0&0&0\\ 
\end{matrix}\right]
\]
Then, the vectors $\{\mathbf{u}_1, \mathbf{u}_2, \mathbf{u}_3\}$ are linearly independent.\\
\textbf{True}. The matrix above is equivalent to the system:
\[
\begin{matrix}
x_1=0\\x_2=0\\x_3=0
\end{matrix}
\]
Since the system has only the trivial solution, the vectors are linearly independent.

\item Let $\{\mathbf{u}_1, \mathbf{u}_2, \mathbf{u}_3\}$ be a set of vectors in $\R^5$, and let
$\mathbf{A}=[\begin{matrix} \mathbf{u}_1&\mathbf{u}_2&\mathbf{u}_3\end{matrix}]$ be the matrix that has these vectors as columns. 
Suppose that the augmented matrix of the system $\mathbf{A}\mathbf{x}=\mathbf{0}$ is equivalent to the following reduced row echelon matrix:
\[
\left[\begin{matrix}
1&0&0&0\\ 
0&0&0&0\\ 
0&0&1&0\\ 
0&0&0&0\\ 
0&0&0&0\\ 
\end{matrix}\right]
\]
Then, the vectors $\{\mathbf{u}_1, \mathbf{u}_2, \mathbf{u}_3, \mathbf{u}_4\}$ are linearly independent.\\
\textbf{False}. The matrix above is equivalent to the system:
\[
\begin{matrix}
x_1=0\\x_3=0
\end{matrix}
\]
Notice that the variable $x_2$ is free, since it does not appear in any of the equations above. So, we can choose any nonzero value for $x_2$ and obtain a nontrivial solution for the system. It follows that the vectors are linearly dependent.

\item Suppose that the homogeneous system $\mathbf{A}\mathbf{x}=\mathbf{0}$ only has the trivial solution. Then, for any vector $\mathbf{b}$, the system $\mathbf{A}\mathbf{x}=\mathbf{b}$ has exactly one solution.\\
\textbf{False}. The system $\mathbf{A}\mathbf{x}=\mathbf{b}$ could have either no solutions or one solution.

\item Suppose that the homogeneous system $\mathbf{A}\mathbf{x}=\mathbf{0}$ has infinitely many solutions. Then, for any vector $\mathbf{b}$, the system $\mathbf{A}\mathbf{x}=\mathbf{b}$ always has infinitely many solutions.\\
\textbf{False}. The system $\mathbf{A}\mathbf{x}=\mathbf{b}$ may have no solutions. (However, if it has solutions, there must be infinitely many of them.)

\item Suppose that the homogeneous system $\mathbf{A}\mathbf{x}=\mathbf{0}$ only has the trivial solution. Then, there may exist vectors $\mathbf{b}$, such that the system $\mathbf{A}\mathbf{x}=\mathbf{b}$ does not have any solutions.\\
\textbf{True}. If $\mathbf{A}\mathbf{x}=\mathbf{0}$ only has the trivial solution, then $\mathbf{A}\mathbf{x}=\mathbf{b}$
may have either zero solutions or exactly one solution.

\end{enumerate}



\end{document}

\sffamily
