\documentclass[12pt]{article}

\input{../../../../../fimacros.tex}

\setheadings{MTH 288 --- Linear Algebra --- Test 1 Information}

\begin{document}

\sffamily

\section{Test Information}

\subsection{Time and Location}

\begin{center}
{\Large
Wednesday, February 17, in the Computer Lab (RT1501)
}
\end{center}

\subsection{Test Topics}

\begin{itemize}
\item Section 1.1: Lines and Linear Equations
\item Section 1.2: Linear Systems and Matrices
\item Section 1.4: Applications of Linear Systems
\item Section 2.1: Vectors
\item Section 2.2: Span
\item Section 2.3: Linear Independence
\end{itemize}

Linear algebra is a discipline that is both \emph{computational} and \emph{conceptual}. To be successful in the test, it is essential to study both aspects of the subject and understand how they are related. The following tips will help you achieve this goal:

\begin{itemize}
\item The main computational tool for the test is the Gaussian Elimination Method for solving linear systems. The main skills required are:
\begin{itemize}
\item Understanding the relationship between a linear system and its corresponding augmented matrix.
\item Understanding the goals of Gaussian Elimination, and being able to recognize matrices in row echelon form and reduced row echelon Form.
\item Understanding the three kinds of row operations, and how to use them to transform a matrix into row echelon/reduced row echelon form.
\item Knowing how to interpret a matrix in row echelon/reduced row echelon form, and how to write the solution of a system from such matrix. Knowing how to determine if a system has zero, one or infinitely many solutions.
\end{itemize}
\item There are two main theoretical concepts in this unit: span and linear independence. The following are the most important points:
\begin{itemize}
\item Understanding the definitions of span and linear independence, and being able to translate the definitions into an equivalent linear system.
\item Knowing how to determine if a vector is in the span of a set of vectors.
\item Knowing how to characterize the span of a set of vectors.
\item Being able to determine of a set of vector spans the whole euclidean space $\R^n$.
\item Knowing how to determine if a set of vectors is linearly independent.
\item Understanding the basic theorems about span and linear independence. In particular, knowing what is the minimum number of vectors that are required to span the whole euclidean space $\R^n$, and what is the maximum number of vectors that can be linearly independent in $\R^n$.
\end{itemize}
\item An important concept that helps connect the ideas of span and linear independence with the computational tools provided by linear systems is the idea of \emph{homogeneous system}. The following are important things to know about homogeneous systems:
\begin{itemize}
\item The notion of \emph{trivial solution}, and the fact that a homogeneous system always has a solution, but may also have infinitely many solutions.
\item The relationship between homogeneous systems and the notion of linear independence.
\item The structure that characterizes the reduced row echelon form of a homogeneous linear system with only the trivial solution.
\item The relationship between a homogeneous system with only the trivial solution and the corresponding non-homogeneous system.
\end{itemize}
\end{itemize}

\section{Preparing For the Test}

\subsection{Preparation Materials}

To be well prepared for the test, you must use all of the following:

\begin{itemize} 
\item Your lecture notes. Notice that the lectures contain content, examples and tips that go beyond the textbook.
\item The textbook. Reading and understanding the relevant sections in the text is essential to do well in the test.
\item Online homework. You are expected to know how to solve all problems in the online homework. They are strong candidates to be present in the test.
\item Computer Lab 1. You will be required to use the computer to perform row operations and convert a matrix to row echelon/reduced row echelon form.
\item The list of review problems provided in the Blackboard site.
\item The list of conceptual questions provided in the Blackboard site.
\end{itemize}

\subsection{Preparation Strategies}

\begin{itemize}
\item Mathematics is a subject that can only be learned by \emph{doing}. Do as much practice problems as your time allows.
\item When studying, be an \emph{active reader}. Take notes as you read the text and the lecture notes, make a summary and/or a set of study cards.
\item Trying to simply memorize definitions, theorems and proofs is known to be a failing strategy to learn mathematics. Instead, it is important to develop an \emph{understanding} of how the several concepts fit together. Many people find it helpful to \emph{write} the concepts, since writing ``slows down'' our thought process to a reflection level that makes it easier to learn all the relationships.
\end{itemize}

\subsection{Test Policies}

\begin{itemize}
\item The test is with closed notes and books.
\item You can use the row operation software provided for the computer labs.
\item \emph{No other software is allowed.}
\item You are not allowed to visit any external web site during the test, or use the computer to communicate with anyone.
\item Calculators are allowed, but are not necessary. In fact, they are not recommended.
\item No scratch paper is allowed during the test. The test will contain enough space to contain all solutions.
\item Answers are to be given in exact form. That is, numbers must be expressed as rationals (fractions), not decimal approximations.
\item Solutions must contain the work that shows how answers were obtained. Solutions with no justification will receive no credit. (Unless where it is specifically said that a justification is not necessary.)
\item For the purpose of grading, all work must be written in the provided test. No printouts of Jupyter notebooks will be accepted. So, make sure you copy to the test writeup all the information you deem necessary for the solution of each problem.
\item Write the final answer of each problem in a manner that is consistent with the question asked. If, for example, a problem asks for the solution of a linear system a matrix will not be accepted as a valid answer.
\item \emph{The instructor will not answer questions during the test}. The only exception to this rule consists of readability issues caused by copying problems. Notice that \emph{being able to understand the statement of test problems is part of the skills being tested, and is each student's exclusive responsibility}.
\end{itemize}

\end{document}
















