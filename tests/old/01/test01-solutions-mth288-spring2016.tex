\documentclass[12pt]{article}

\input{../../../../fimacros.tex}

\setheadings{MTH 288 --- Linear Algebra --- Test 1 Solutions}

\begin{document}

\textbf{Problem 1.} (42 points.) Answer the following items based on the linear system below:
\begin{alignat*}{5}
 2x_1 & {}+{} &  6x_2  & {}-{} & 9x_3  & {}-{} & 4x_4 & {}={} &  0 \\
-3x_1 & {}-{} &  11x_2 & {}+{} & 9x_3  & {}-{} &  x_4 & {}={} &  0 \\
  x_1 & {}+{} &   4x_2 & {}-{} & 2x_3  & {}+{} &  x_4 & {}={} &  0
\end{alignat*}

\textbf{(a)} (8 points.) Write the augmented matrix corresponding to this system.

\emph{Solution}.
\[
\left[\begin{matrix}2 & 6 & -9 & -4 & 0\\-3 & -11 & 9 & -1 & 0\\1 & 4 & -2 & 1 & 0\end{matrix}\right]
\]

\bigskip
\textbf{(b)} (16 points) Use a sequence of elementary row operations to find a matrix in reduced row echelon form that is equivalent to the matrix in Part (a). Write, in the space provided below, all row operations and intermediate matrices in your computations.

\emph{Solution}

Starting matrix:
\[
\left[\begin{matrix}2 & 6 & -9 & -4 & 0\\-3 & -11 & 9 & -1 & 0\\1 & 4 & -2 & 1 & 0\end{matrix}\right]
\]

\textbf{(1)} \texttt{R1<=>R3}
\[\left[\begin{matrix}1 & 4 & -2 & 1 & 0\\-3 & -11 & 9 & -1 & 0\\2 & 6 & -9 & -4 & 0\end{matrix}\right]\]

\textbf{(2)} \texttt{R1*(3)+R2=>R2, R1*(-2)+R3=>R3}
\[\left[\begin{matrix}1 & 4 & -2 & 1 & 0\\0 & 1 & 3 & 2 & 0\\0 & -2 & -5 & -6 & 0\end{matrix}\right]\]

\textbf{(3)} \texttt{R2*(2)+R3=>R3, R2*(-4)+R1=>R1}
\[\left[\begin{matrix}1 & 0 & -14 & -7 & 0\\0 & 1 & 3 & 2 & 0\\0 & 0 & 1 & -2 & 0\end{matrix}\right]\]

\textbf{(4)} \texttt{R3*(14)+R1=>R1, R3*(-3)+R2=>R2}
\[\left[\begin{matrix}1 & 0 & 0 & -35 & 0\\0 & 1 & 0 & 8 & 0\\0 & 0 & 1 & -2 & 0\end{matrix}\right]\]


\textbf{(c)} (8 points.) Write the linear system that corresponds to the reduced row echelon form matrix from the previous item.

\emph{Solution}.
\begin{alignat*}{4}
x_1 & {}-{} & 35x_4 & {}={} & 0\\
x_2 & {}+{} & 8x_4  & {}={} & 0\\
x_3 & {}-{} & 2x_4  & {}={} & 0
\end{alignat*}

\textbf{(d)} (10 points) Determine the set of solution of the system. If there are infinitely many solutions, write the solutions in parametric form, that is, using a set of independent variables ($s_1$, \dots, etc.) to express the values of $x_1$, $x_2$, $x_3$ and $x_4$. Finally, determine if the system has zero, one or infinitely many solutions.

\emph{Solution}.
\begin{align*}
x_1 &= 35s\\
x_2 &= -8s\\
x_3 &=  2s\\
x_4 &= s
\end{align*}
where $s$ is an arbitrary scalar. There are infinitely many solutions.

\textbf{Problem 2.} (10 points.)
When propane burns in oxygen, it produces carbon dioxide and water:
\[
\text{C}_3\text{H}_8 + \text{O}_2 \longrightarrow \text{C}\text{O}_2+\text{H}_2\text{O}
\]
We want to balance this chemical equation. Write a system of linear equations that can be used to solve this problem, using the following variables:
\begin{align*}
x_1 &= \text{(amount of propane $\text{C}_3\text{H}_8$ in the reagents)}\\
x_2 &= \text{(amount of oxygen $\text{O}_2$ in the reagents})\\
x_3 &= \text{(amount of carbon dioxide $\text{C}\text{O}_2$ in the products)}\\
x_4 &= \text{(amount of water $\text{H}_2\text{O}$ in the products)}
\end{align*}

\emph{Solution}.

Equation for carbon (C): $3x_1=x_3$, or $3x_1-x_3=0$.

Equation for hydrogen (H): $8x_1=2x_4$, or $8x_1-2x_4=0$.

Equation for oxygen (O): $2x_2=2x_3+x_4$, or $2x_2-2x_3-x_4=0$

We get the following system:
\begin{alignat*}{5}
3x_1 & {}{} &      & {}-{} &  x_3 & {}{}  &  & {}={} & 0\\
8x_1 & {}{} &      & {}{}  &      & {}-{} & 2x_4 & {}={} & 0\\
     & {}{} & 2x_2 & {}-{} & 2x_3 & {}-{} &  x_4 & {}={} & 0
\end{alignat*}

\textbf{Problem 3.} (18 points.) Determine if each of the following statements is true or false, and provide a brief justification for your answer.

\textbf{(a)} (3 points.) A linear system with 3 equations and 3 unknowns always has exactly one solution.

\emph{Solution}. False.

Justification: A linear system with the same number of equations and unknowns can have zero, one or infinitely many solutions.

\textbf{(b)} (3 points.) If a homogeneous system has 5 equations and 8 unknowns, then it must have infinitely many solutions.

\emph{Solution} True.

Justification: A homogeneous system always has at least one solution (the trivial solution). The augmented matrix for this system has 5 rows and 9 columns. The equivalent reduced row echelon matrix will have at least 3 columns with no leading terms, so there are at least 3 free variables that can be chosen arbitrarily, so there will be infinitely many solutions.

\textbf{(c)} (3 points.) The matrix below is in reduced row echelon form:
\[
\left[
\begin{matrix}
1 & 2 & 0 & -1 & 0 & 4\\
0 & 0 & 1 &  1 & 0 & -2\\
0 & 0 & 0 & 0 & 1 & 0 \\
0 & 0 & 0 & 0 & 0 & 0
\end{matrix}
\right]
\]

\emph{Solution}. True

Justification: The following conditions are all satisfied:
\begin{itemize}
\item Rows 1, 2 and 3 have a leading nonzero term equal to 1, located at columns 1, 3 and 5, respectively.
\item The leading nonzero term in each row is strictly to the right to the leading term in the previous row.
\item The other entries in columns with a leading nonzero term are zero.
\item All zero rows are at the bottom of the matrix.
\end{itemize}

\textbf{(d)} (3 points.) If a set of vectors $\{\mathbf{u}_1,\mathbf{u}_2,\mathbf{u}_3,\mathbf{u}_4\}$ are linearly dependent and $\mathbf{u}_5$ is an arbitrary vector, then the vectors  $\{\mathbf{u}_1,\mathbf{u}_2,\mathbf{u}_3,\mathbf{u}_4,\mathbf{u}_5\}$ also are linearly dependent.

\emph{Solution}. True

Justification: Suppose $\{\mathbf{u}_1,\mathbf{u}_2,\mathbf{u}_3,\mathbf{u}_4\}$ are linear dependent. Then, there are $c_1$, $c_2$, $c_3$ and $c_4$, not all zero, such that:
\[
c_1\mathbf{u}_1+c_2\mathbf{u}_2+c_3\mathbf{u}_3+c_4\mathbf{u}_4=\mathbf{0}
\]
Then, the following is a nontrivial linear combination of $\{\mathbf{u}_1,\mathbf{u}_2,\mathbf{u}_3,\mathbf{u}_4,\mathbf{u}_5\}$ that results in the zero vector:
\[
c_1\mathbf{u}_1+c_2\mathbf{u}_2+c_3\mathbf{u}_3+c_4\mathbf{u}_4+0\mathbf{u}_5=\mathbf{0}
\]

\textbf{(e)} (3 points.) A set of 8 vectors can span $\R^5$.

\emph{Solution}. True

Justification: To span $\R^5$ we need at least 5 vectors.

\textbf{(f)} (3 points.) Any set of 8 vectors spans $\R^5$.

\emph{Solution} False

Justification: If, for example, we let $\mathbf{u}$ be any nonzero vector in $\R^5$, the set of vectors 
\[\{\mathbf{u},2\mathbf{u},3\mathbf{u},4\mathbf{u},5\mathbf{u},6\mathbf{u},7\mathbf{u},8\mathbf{u}\}\]
does not span $\R^5$, since a linear combination of these vectors is always a scalar multiple of $\mathbf{u}$.

\clearpage

\textbf{Problem 4.} (20 points.) To answer the following items, consider the following 3 vectors in $\R^3$:

\[
\textbf{u}_1=\left[\begin{matrix}  1 \\ 2\\ -3\end{matrix}\right]\quad
\textbf{u}_2=\left[\begin{matrix}  3 \\ 4\\  2\end{matrix}\right]\quad
\textbf{u}_3=\left[\begin{matrix} 7 \\ 8\\  12\end{matrix}\right]
\]

\textbf{(a)} (10 points.) Determine if these vectors are linearly independent in $\R^3$. Show all computations, and explain your solution in terms of the definition of linear independence.

\emph{Solution}. We have to determine if the following system has only the trivial solution:
\[
x_1\left[\begin{matrix}  1 \\ 2\\ -3\end{matrix}\right]+
x_2\left[\begin{matrix}  3 \\ 4\\  2\end{matrix}\right]+
x_3\left[\begin{matrix} 7 \\ 8\\ 12\end{matrix}\right]=
\left[\begin{matrix}0\\0\\0\end{matrix}\right]
\]

The augmented matrix of this system is:

\[\left[\begin{matrix}1 & 3 & 7 & 0\\2 & 4 & 8 & 0\\-3 & 2 & 12 & 0\end{matrix}\right]\]

Gaussian elimination gives:

\textbf{(1)} \texttt{R1*(-2)+R2=>R2, R1*(3)+R3=>R3}
\[\left[\begin{matrix}1 & 3 & 7 & 0\\0 & -2 & -6 & 0\\0 & 11 & 33 & 0\end{matrix}\right]\]

\textbf{(2)} \texttt{R2*(-1/2)=>R2}
\[\left[\begin{matrix}1 & 3 & 7 & 0\\0 & 1 & 3 & 0\\0 & 11 & 33 & 0\end{matrix}\right]\]

\textbf{(3)} \texttt{R2*(-3)+R1=>R1, R2*(-11)+R3=>R3}
\[\left[\begin{matrix}1 & 0 & -2 & 0\\0 & 1 & 3 & 0\\0 & 0 & 0 & 0\end{matrix}\right]\]

This is in reduced row echelon form, and corresponds to the system:
\begin{alignat*}{4}
x_1 & {}{} &     & {}-{} & 2x_3 & {}={} & 0\\
    & {}{} & x_2 & {}+{} & 3x_3 & {}={} & 0
\end{alignat*}
This system has infinitely many solutions:
\begin{align*}
x_1 &=  2s\\
x_2 &= -3s\\
x_3 &=   s
\end{align*}
We conclude that the system has nontrivial solutions, so the vectors are \emph{linearly dependent}. If we let, for example, $s=1$, we get $x_1=2$, $x_2=-3$, $x_3=1$, so that the following is a nonlinear combination of the given vectors that is equal to the zero vector:
\[
2\left[\begin{matrix}  1 \\ 2\\ -3\end{matrix}\right]
-3\left[\begin{matrix}  3 \\ 4\\  2\end{matrix}\right]+
\left[\begin{matrix} 7 \\ 8\\ 12\end{matrix}\right]=
\left[\begin{matrix}0\\0\\0\end{matrix}\right]
\]

\textbf{(b)} (10 points.) Determine if the vector
\[
\textbf{v}=\left[\begin{matrix}  1 \\ 4\\ -14\end{matrix}\right]
\]
is in the span of the vectors $\{\mathbf{u}_1,\mathbf{u}_2,\mathbf{u}_3\}$. Show all computations, and explain your solution in terms of the definition of span.

\emph{Solution}. For this part, we need to consider the system:
\[
x_1\left[\begin{matrix}  1 \\ 2\\ -3\end{matrix}\right]+
x_2\left[\begin{matrix}  3 \\ 4\\  2\end{matrix}\right]+
x_3\left[\begin{matrix} 7 \\ 8\\ 12\end{matrix}\right]=
\left[\begin{matrix}1\\4\\-14\end{matrix}\right]
\]

The augmented matrix is now:
\[\left[\begin{matrix}1 & 3 & 7 & 1\\2 & 4 & 8 & 4\\-3 & 2 & 12 & -14\end{matrix}\right]\]

If we do the same sequence of row operations as in the previous item, we get the reduced row echelon matrix:
\[\left[\begin{matrix}1 & 0 & -2 & 4\\0 & 1 & 3 & -1\\0 & 0 & 0 & 0\end{matrix}\right]\]

This corresponds to the system:
\begin{alignat*}{4}
x_1 & {}{} &     & {}-{} & 2x_3 & {}={} & 4\\
    & {}{} & x_2 & {}+{} & 3x_3 & {}={} & -1
\end{alignat*}
This system has infinitely many solutions:
\begin{align*}
x_1 &= 4 + 2s\\
x_2 &= -1 - 3s\\
x_3 &=   s
\end{align*}
Choosing any value of $s$ yields a linear combination of the three given vectors that is equal to $\mathbf{v}$. For example, letting $s=0$ we get $x_1=4$, $x_2=-1$ and $x_3=0$, and we get the following linear combination.
\[
4\left[\begin{matrix}  1 \\ 2\\ -3\end{matrix}\right]
-\left[\begin{matrix}  3 \\ 4\\  2\end{matrix}\right]+
0\left[\begin{matrix} 7 \\ 8\\ 12\end{matrix}\right]=
\left[\begin{matrix}1\\4\\-14\end{matrix}\right]
\]


\textbf{Problem 5.} (10 points.) Suppose that $\mathbf{v}$ is in the span of the vectors $\{\mathbf{u_1},\mathbf{u}_2\}$. In other words, we are assuming that $\mathbf{v}$ can be written as a linear combination of $\mathbf{u}_1$ and $\mathbf{u}_2$:
\[
\mathbf{v}=c_1\mathbf{u}_1+c_2\mathbf{u}_2.
\]

Show that $\mathbf{v}$ is in the span of $\{\mathbf{u}_1+\mathbf{u}_2, \mathbf{u}_1-\mathbf{u}_2\}$.

\textbf{Solution} To show that $\mathbf{v}$ is in the span of $\{\mathbf{u}_1+\mathbf{u}_2, \mathbf{u}_1-\mathbf{u}_2\}$, we have to solve the equation:
\[
x_1(\mathbf{u}_1+\mathbf{u}_2)+x_2(\mathbf{u}_1-\mathbf{u}_2)=\mathbf{v}
\]
This equation can be rewritten as:
\[
(x_1+x_2)\mathbf{u}_1+(x_1-x_2)\mathbf{u}_2=\mathbf{v}
\]
We are given that:
\[
\mathbf{v}=c_1\mathbf{u}_1+c_2\mathbf{u}_2
\]
Comparing the last two equations, we see that we need:
\begin{alignat*}{3}
x_1 & {}+{} & x_2 {}={} c_1\\
x_1 & {}-{} & x_2 {}={} c_2
\end{alignat*}
To solve this system, we can add the two equations to get:
\[
2x_1=c_1+c_2\quad\text{or}\quad x_1=\frac{c_1+c_2}{2}
\]
Subtracting the equations we get:
\[
2x_2=c_1-c_2\quad\text{or}\text x_2=\frac{c_1-c_2}{2}
\]
It follows that we can write $\mathbf{v}$ as a linear combination of $\mathbf{u}_1+\mathbf{u}_2$ and $\mathbf{u}_1-\mathbf{u}_2$:
\[
\left(\frac{c_1+c_2}{2}\right)(\mathbf{u}_1+\mathbf{u}_2)+\left(\frac{c_1-c_2}{2}\right)(\mathbf{u}_1-\mathbf{u}_2)=\mathbf{v}
\]

\end{document}
















