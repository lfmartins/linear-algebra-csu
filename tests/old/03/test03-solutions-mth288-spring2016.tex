\documentclass[12pt]{article}

\input{../../../../fimacros.tex}

\setheadings{MTH 288 --- Linear Algebra --- Test 3 Solutions}

\begin{document}

% Eigenvalues and eigenspaces
\textbf{Problem 1.}(30 points.) Let $A$ be the matrix:
\[
A=\left[\begin{matrix}-19 & -10 & 20\\-5 & -14 & 10\\-15 & -15 & 21\end{matrix}\right]
\]

\textbf{(a)} (15 points) Find all the eigenvalues of $A$, and determine the multiplicity of each eigenvalue.

\emph{Solution}. First, compute the eigenvalues:
\[
\det(A-\lambda I)=\det \left[\begin{matrix}- \lambda - 19 & -10 & 20\\-5 & - \lambda - 14 & 10\\-15 & -15 & - \lambda + 21\end{matrix}\right]=- \lambda^{3} - 12 \lambda^{2} + 27 \lambda + 486=- \left(\lambda - 6\right) \left(\lambda + 9\right)^{2}
\]
We conclude that the eigenvalues are $\lambda_1=-9$, with multiplicity 2 and $\lambda_2=6$, with multiplicity 1.\proofend

\textbf{(b)} (15 points) Find a basis for the eigenspace associated to each of the eigenvalues of $A$.

\emph{Solution}.

Eigenspace of $\lambda_1=-9$:
\[
A-(-9)I = \left[\begin{matrix}-10 & -10 & 20\\-5 & -5 & 10\\-15 & -15 & 30\end{matrix}\right]\sim\left[\begin{matrix}1 & 1 & -2\\0 & 0 & 0\\0 & 0 & 0\end{matrix}\right]
\]
From the reduced row echelon form, the equations for a vector in the kernel are:
\[
x_1+x_2-2x_3=0
\]
Since this system has two free variables, the kernel has dimension $2$.

Letting $x_2=1$, $x_3=0$ we get the vector
\[
\begin{bmatrix}-1\\1\\0\end{bmatrix}
\]

Letting $x_2=0$ and $x_3=1$ we get the vector:
\[
\begin{bmatrix}2\\0\\1\end{bmatrix}
\]

We conclude that a basis of the eigenspace of $\lambda_1=-9$ is:
\[
\left\{
\begin{bmatrix}-1\\1\\0\end{bmatrix},
\begin{bmatrix}2\\0\\1\end{bmatrix}
\right\}
\]

Eigenspace of $\lambda_2=6$:
\[
A-6I=
\left[\begin{matrix}-25 & -10 & 20\\-5 & -20 & 10\\-15 & -15 & 15\end{matrix}\right]\sim
\left[\begin{matrix}1 & 0 & - \frac{2}{3}\\0 & 1 & - \frac{1}{3}\\0 & 0 & 0\end{matrix}\right]
\]
The equations for a vector in the kernel are:
\[
x_1-\frac{2}{3}x_3=0,\quad x_2-\frac{1}{3}x_3=0
\]
There is only one free variable, so the kernel has dimension $1$. Choosing $x_3=3$ we get the following basis for the kernel:
\[
\left\{
\begin{bmatrix}2\\1\\3\end{bmatrix}
\right\}
\]
\proofend

\textbf{Problem 2.} (30 points.) Answer the following items for the following basis of $\R^{3}$:
\[
{\cal B}=\left\{
\begin{bmatrix} 1\\ 0\\ 0\end{bmatrix},
\begin{bmatrix} -1\\ 2\\ 1\end{bmatrix},
\begin{bmatrix} 1\\ -1\\ 0\end{bmatrix}
\right\}
\]

\textbf{(a)} (15 points) Convert the following vector to a coordinate vector with respect to the standard basis: 
\[
[\mathbf{x}]_{\cal B}=\begin{bmatrix}2\\-1\\3\end{bmatrix}
\]

\emph{Solution}. Let $P$ be the change of basis matrix:
\[
P=\left[\begin{matrix}1 & -1 & 1\\0 & 2 & -1\\0 & 1 & 0\end{matrix}\right]
\]
Then:
\[
\mathbf{x}=P[\mathbf{x}]_{\cal B}=\left[\begin{matrix}1 & -1 & 1\\0 & 2 & -1\\0 & 1 & 0\end{matrix}\right]
\left[\begin{matrix}2\\-1\\3\end{matrix}\right]=\left[\begin{matrix}6\\-5\\-1\end{matrix}\right]
\]
\proofend


\textbf{(b)} (15 points) Convert the following vector to a coordinate vector with respect to the basis ${\cal B}$:
\[
\mathbf{x}=\begin{bmatrix}-2\\1\\2\end{bmatrix}
\]

\emph{Solution}. We need the inverse of the matrix $P$:
\[
P^{-1}=\left[\begin{matrix}1 & 1 & -1\\0 & 0 & 1\\0 & -1 & 2\end{matrix}\right]
\]
Then:
\[
[\mathbf{x}]_{\cal B}=P^{-1}\mathbf{x}=
\left[\begin{matrix}1 & 1 & -1\\0 & 0 & 1\\0 & -1 & 2\end{matrix}\right]
\left[\begin{matrix}-2\\1\\1\end{matrix}\right]=\left[\begin{matrix}-2\\1\\1\end{matrix}\right]
\]
\proofend


\textbf{Problem 3.} (30 points.) Let $A$ be a $4\times 4$ matrix. The following information is given about the eigenvalues and corresponding eigenspaces of the matrix $A$:
\[
\text{Eigenvalue: }\lambda_1=1;\quad
\text{Basis for eigenspace: } \left\{\begin{bmatrix} 2\\ 0\\ 1\\ 1\end{bmatrix},
\begin{bmatrix} 0\\ 1\\ 2\\ 1\end{bmatrix}\right\}
\]

\[
\text{Eigenvalue: }\lambda_2=-2;\quad
\text{Basis for eigenspace: } \left\{\begin{bmatrix} 0\\ 1\\ 3\\ 1\end{bmatrix},
\begin{bmatrix} 3\\ 1\\ 4\\ 1\end{bmatrix}\right\}
\]

Is it possible to determine the matrix $A$ using only this information? If possible, find the matrix $A$. If not, explain why.

\emph{Solution}. Since there are four linearly independent eigenvectors, there is a basis of $\R^4$ consisting only of eigenvectors, and the matrix $A$ is diagonalizable. This means that, if we let:
\[
P=\left[\begin{matrix}2 & 0 & 0 & 3\\0 & 1 & 1 & 1\\1 & 2 & 3 & 4\\1 & 1 & 1 & 1\end{matrix}\right]
\quad\text{and}\quad
D=\left[\begin{matrix}1 & 0 & 0 & 0\\0 & 1 & 0 & 0\\0 & 0 & -2 & 0\\0 & 0 & 0 & -2\end{matrix}\right]
\]
we have $P^{-1}AP=D$. So:
\[
A = PDP^{-1}=\left[\begin{matrix}-2 & -6 & 0 & 6\\1 & 6 & -3 & 1\\2 & 13 & -8 & 5\\1 & 5 & -3 & 2\end{matrix}\right]
\]
\proofend

\textbf{Problem 4.} (10 points.) Let $a$ be a scalar and consider the matrix:
\[
A = \begin{bmatrix}a&1\\0&a\end{bmatrix}
\]
Answer the following items for the matrix $A$. Notice that your answers will depend on the unspecified scalar $a$

\textbf{(a)} Determine the eigenvalues of the matrix $A$, and specify the multiplicity of each eigenvalue.

\emph{Solution}.
\[
\det(A-\lambda I)=\det\begin{bmatrix}a-\lambda & 1\\0&a-\lambda\end{bmatrix}=(a-\lambda)^2
\]
Then only eigenvalue is $\lambda=a$, which has multiplicity $2$\proofend

\textbf{(b)} Find a basis for the eigenspace corresponding to each of the eigenvalues you found in the previous item.

\emph{Solution}.
\[
A-aI=\begin{bmatrix}0&1\\0&0\end{bmatrix}
\]
This matrix is already in reduced row echelon form, so the equations for the kernel are:
\[
x_2=0
\]
There is only one free variable, namely $x_1$, so a basis for eigenspace of $\lambda=a$ is obtained by choosing $x_1=1$:
\[
\left\{\begin{bmatrix}1\\0\end{bmatrix}\right\}
\]
\proofend

\textbf{(c)} Determine if the matrix $A$ is diagonalizable or not, and justify your answer.

\emph{Solution}. Since there is only one linearly independent eigenvector, it is not possible to find a basis of $A$ that consists only of eigenvectors, and the matrix is not diagonalizable.\proofend

\end{document}