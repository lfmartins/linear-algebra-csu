\documentclass[12pt]{article}

\input{../../../../fimacros.tex}

\setheadings{MTH 288 --- Linear Algebra --- Test 2}

\begin{document}

Name and Student ID: \hrulefill


\textbf{Instructions.} All solutions must be justified, unless otherwise stated. Show all work leading to your answer in each problem. Solutions without appropriate work that supports it will receive no credit. All work must be written in the test. Do not attach computer printouts to the test. If not enough space is provided for an answer, continue it in the back of the page.

Please identify your final answer to each problem by surrounding it with a rectangle.


\clearpage


% Basis for subspace
\textbf{Problem 1.}(20 points.) 
Find a basis for the subspace of $\R^3$ spanned by the vectors:

$$
\mathbf{u}_1=\left[\begin{matrix}1\\-1\\2\end{matrix}\right]\quad
\mathbf{u}_2=\left[\begin{matrix}-4\\4\\-8\end{matrix}\right]\quad
\mathbf{u}_3=\left[\begin{matrix}2\\0\\-3\end{matrix}\right]\quad
\mathbf{u}_4=\left[\begin{matrix}-4\\-2\\13\end{matrix}\right]\quad
$$
\clearpage

\textbf{Problem 2.}(20 points.)
Answer the following items for the matrix:
\[
A=
\begin{bmatrix}1&3&2&0\\3&11&7&1\\1&1&4&0\end{bmatrix}
\] 

\textbf{(a)} (10 points.) Find a basis for the range of $A$, and determine the dimension of the range.

\vfill
(Problem continues on next page).
\clearpage

\textbf{(b)} (10 points.) Find a basis for the kernel of $A$, and determine the dimension of the kernel.

\clearpage

\textbf{Problem 3.} (16 points.) Find scalars $a$ and $b$ that make the following matrix identity true.

\[
\left[\begin{matrix}a & 1 & 2\\2 & 2 & b\\-1 & 4 & 1\end{matrix}\right]
\left[\begin{matrix}2 & 2\\1 & -1\\3 & -2\end{matrix}\right]=
\left[\begin{matrix}8 & -4\\11 & - \frac{4}{3}\\5 & -8\end{matrix}\right]
\] 

\clearpage

\textbf{Problem 4.} (16 points.) Find the determinants for the following matrices. Explain the method you used to find the determinant, showing all computations. Do not use Python for this problem.

\textbf{(a)} (8 points.)
$
\det\begin{bmatrix}1&0&3\\2&-2&1\\3&4&1\end{bmatrix}
$

\vskip4in

\textbf{(b)} (8 points.)
$
\det\begin{bmatrix}
2&3&-5&9&0\\
0&-1&3&-4&2\\
0&0&3&7&-11\\
0&0&0&\frac{1}{2}&-1\\
0&0&0&0&10\\
\end{bmatrix} 
$
\clearpage

\textbf{Problem 5.} (18 points.) Determine if each of the following statements is true or false, and provide a brief justification for your answer.

\textbf{(a)} (3 points.) A linear transformation $T:\R^8\to\R^4$ must be onto.

Circle one:\ \ \ \ \  True\ \ \ \ \ False

Justification:

\vskip1.5in


\textbf{(b)} (3 points.) A linear transformation $T:\R^8\to\R^4$ can be one-to-one.

Circle one:\ \ \ \ \  True\ \ \ \ \ False

Justification:

\vskip1.5in


\textbf{(c)} (3 points.) There is exactly one value of $a$ for which the matrix below is singular:
\[
\begin{bmatrix}a&2\\2&a\end{bmatrix}
\]

Circle one:\ \ \ \ \  True\ \ \ \ \ False

Justification:

\clearpage

\textbf{(d)} (3 points.) The two matrices below have the same determinant.
\[
\begin{bmatrix}a&b&c\\d&f&g\\h&i&j\end{bmatrix}\quad
\begin{bmatrix}a&b+2a&c\\d&f+2d&g\\h&i+2h&j\end{bmatrix}
\]

Circle one:\ \ \ \ \  True\ \ \ \ \ False

Justification:

\vskip1.5in


\textbf{(e)} (3 points.) If $\mathbf{u}_1$, $\mathbf{u}_2$ and $\mathbf{u}_3$ are distinct vectors in $\R^4$, then the subspace spanned by $\{\mathbf{u}_1,\mathbf{u}_2,\mathbf{u}_3\}$ has dimension 3.

Circle one:\ \ \ \ \  True\ \ \ \ \ False

Justification:

\vskip1.5in

\textbf{(e)} (3 points.) If all entries of a matrix are positive, then the matrix is invertible.

Circle one:\ \ \ \ \  True\ \ \ \ \ False

Justification:

\clearpage

\textbf{Problem 6.} (10 points.) Find the area of the parallelogram in $\R^2$ determined by the vectors
\[
\begin{bmatrix}2\\3\end{bmatrix}\quad\text{and}\quad\begin{bmatrix}-2\\4\end{bmatrix}
\]

\end{document}