\documentclass[12pt]{article}

\input{../../../../../fimacros.tex}

\setheadings{MTH 288 --- Linear Algebra --- Test 2 Concepts}

\begin{document}

\section{Questions}
For each of the following statements determine if it is true or false, and provide a brief justification for your answer.

It is \emph{guaranteed} that one or more of these questions will be in the test, in \emph{exactly the same form} as they appear below.


\begin{enumerate}

\item The set of solutions of the linear system:
\[
\begin{bmatrix}1&2&-1\\2&3&5\\-4&-2&3\end{bmatrix}
\begin{bmatrix}x_1\\x_2\\x_3\end{bmatrix}=
\begin{bmatrix}0\\0\\0\end{bmatrix}
\]
is a linear subspace of the euclidean vector space $\R^3$.

\item If $T:\R^3\to\R^4$ is a linear transformation, then $T$ is one-to-one.

False. The identically zero transformation, $T(\mathbf{x})=0$ for all $\mathbf{x\in\R^3}$ is not one-to-one.

\item If $T:\R^4\to\R^3$ is a linear transformation, then $T$ is onto.

False. The identically zero transformation, $T(\mathbf{x})=0$ for all $\mathbf{x\in\R^4}$ is not onto.

\item If $T:\R^8\to\R^5$ is a linear transformation, then $T$ can be onto.

True. The range of $T$ is a linear subspace of dimension at most 8. If the dimension of the range is $5$, the transformation must be onto. For a concrete example, consider the transformation:
\[
T(\mathbf{x})=
\begin{bmatrix}
1&0&0&0&0&0&0&0\\
0&1&0&0&0&0&0&0\\
0&0&1&0&0&0&0&0\\
0&0&0&1&0&0&0&0\\
0&0&0&0&1&0&0&0\\
\end{bmatrix}
\begin{bmatrix}
x_1\\x_2\\x_3\\x_4\\x_5\\x_6\\x_7\\x_8\\
\end{bmatrix}=
\begin{bmatrix}
x_1\\x_2\\x_3\\x_4\\x_5\\
\end{bmatrix}
\]

\item If $T:\R^8\to\R^5$ is a linear transformation, then $T$ can be one-to-one.

False. The range must be a subspace with dimension at most 5. Since the domain has dimension 8, which is larger than 5, the transformation can't be one-to-one.

\item If $T:\R^4\to\R^4$ is a linear transformation, then $T$ can be invertible.

True. If both domain and codomain have the same dimension, the transformation can be invertible.

\item If a linear transformation $T:\R^3\to\R^3$ is one-to-one, then $T$ must also be onto.

True. If $T$ is one-to-one, the dimension of the range equal do the dimension of the domain, which is $3$. So, the range is a subspace of dimension 3 of $\R^3$, and must be all of $\R^3$.

\item If $A$ and $B$ are $n\times n$ matrices, then $(A+B)(A-B)=A^2-B^2$.

False. $(A+B)(A-B)=A^2+BA-AB-B^2$, and the middle terms don't cancel because matrix multiplication is not commutative.

\item If $A$ is an $n\times n$ matrix and $A^3=0$, then $A=0$.

False. Let:
\[
A=
\begin{bmatrix}
0&1&0\\0&0&1\\0&0&0
\end{bmatrix}.
\]
Then, $A$ is not zero but $A^3$ is equal to the zero matrix.

\item If $A$ and $B$ are $n\times n$ matrices, then $(AB)^T=B^TA^T$

True. This is a theorem from the book.

\item If $A$ is an invertible $n\times n$ matrix, then the number of solutions to $A\mathbf{x}=\mathbf{b}$ is always greater than zero.

True. The system always has exactly one solution, namely $\mathbf{x}=A^{-1}\mathbf{b}$.

\item If the columns of an $n \times n$ matrix $A$ span $\R^n$, then $A$ is nonsingular.

True. The space spanned by the columns is equal to the range of the linear transformation $T(\mathbf{x})=A\mathbf{x}$. Thus, if the columns of $A$ span $\R^5$, then $T$ is onto. Since the dimensions of the domain and the range are the same, $T$ must also be one-to-one, so it is invertible.

\item If $A$ and $B$ are invertible, then $(AB)^{-1}=A^{-1}B^{-1}$.

False. The correct formula is $(AB)^{-1}=B^{-1}A^{-1}$, which is not equal to $A^{-1}B^{-1}$ because matrix multiplication is not commutative.

\item The matrix
\[
\begin{bmatrix}2&3\\4&6\end{bmatrix}
\]
is invertible.

False, because:
\[
\det\begin{bmatrix}2&3\\4&6\end{bmatrix} = 12-12=0.
\]

\item If $A$ is a $5\times 3$ matrix, then $\text{null}\,(A)$ forms a subspace of $\R^5$.

False. The linear transformation $T(\mathbf{x})=A\mathbf{x}$ takes vectors from $\R^3$ to vectors in $\R^5$. The kernel of a transformation is a subspace of its \emph{domain}, which in this case is $\R^3$.

\item If $A$ is a $5\times 3$ matrix, then $\text{null}\,(A)$ forms a subspace of $\R^3$.

True. Viewing $A$ as defining a linear transformation, as in the previous question, the kernel is a subspace of the domain, which is $\R^3$.

\item If $T:\R^5\to\R^8$ is a linear transformation, then $\text{range}\,(T)$ forms a subspace of $\R^8$.

True. The range is a subspace of the codomain, which in this case is $\R^8$.

\item The dimension of a subspace of $\R^5$ is one of the integers $1$, $2$, $3$ or $4$.

False. The dimension can also e $0$ or $5$, corresponding, respectively, to the null subspace consisting only of the zero vector and the whole space.

\item Let $\{\mathbf{u}_1,\mathbf{u}_2,\mathbf{u}_3\}$ be any set of $3$ vectors in $\R^5$. Then, it is always possible to add $2$ vectors to the set to form a basis of $\R^5$.

False. If the vectors $\mathbf{u}_1,\mathbf{u}_2,\mathbf{u}_3$ are linearly dependent, it is not possible to extend the set to form a basis.

\item The only subspace of $\R^8$ that has dimension 8 is $\R^8$ itself.

True. $\R^8$ has dimension $8$ itself, so any subspace of dimension $8$ must be equal to $\R^8$.

\item $\R^3$ has infinitely many subspaces of dimension $2$.

True. A subspace of dimension $2$ can be identified as a plane through the origin, and there clearly are infinitely many of those planes.

\item The rank of a matrix cannot exceed the number of rows of $A$.

True. The rank of a matrix is the dimension of the row space of the matrix, which is smaller than the number of rows.

\item If $A$ is not a square matrix, than the dimension of the row space of $A$ is different of the dimension of the column space of $A$.

False. According to a theorem in the book, both the row space and the column space have the same dimension, which is equal to the rank.

\item If the system $A\mathbf{x}=\mathbf{b}$ has solutions, than $\mathbf{b}$ is in the row space of $A$.

False. If the system has solutions, then $\mathbf{b}$ is in the \emph{column} space of $A$.

\item If $A$ is an $n\times n$ matrix with all positive entries, then $\det(A)$ is positive.

False. Consider the counterexample:
\[
\det
\begin{bmatrix}1&4\\1&1\end{bmatrix}=-3.
\]

\item If $A$ is an upper triangular square matrix, then $\det(A)\ne0$.

False. The determinant of an upper triangular matrix is the product of the diagonal entries. So, if one of the values on the diagonal is zero, then the determinant is zero.

\item If $A$ and $B$ are $n\times n$ matrices, then $\det(A+B)=\det(A)+\det(B)$.

False. Let:
\[
A=\begin{bmatrix}1&2\\2&1\end{bmatrix}\quad\text{and}\quad B=\begin{bmatrix}0&1\\1&0\end{bmatrix}
\]
Then, $\det(A)=1-4=-3$ and $\det(B)=0-1=-1$. So, $\det(A)+\det(B)=-4$. On the other hand,
\[
\det(A+B)=\det\begin{bmatrix}1&3\\3&1\end{bmatrix}=1-9=-8.
\]

\item If $A$ is an $n\times n$ matrix and $c$ is a scalar, then $\det(cA)=c^n\det(A)$

True. When we multiply a row of a matrix by a scalar $c$, the determinant is multiplied by $c$. When we compute $cA$, all $n$ rows are multiplied by $c$, so the determinant is multiplied by $c^n$.

\item If $a$ and $b$ are real numbers that are not both zero, then the matrix
\[
\begin{bmatrix}a&-b\\b&a\end{bmatrix}
\]
is nonsingular.

True. The determinant of the matrix is:
\[
\det\begin{bmatrix}a&-b\\b&a\end{bmatrix}=a\cdot a- b\cdot(-b)=a^2+b^2
\]
Since $a$ and $b$ are not both zero, the quantity above is always positive. Since the determinant is nonzero, the matrix is always invertible.

\item Let $A$ be an $n\times n$ matrix such that $\det(A)\ne 0$. Then the columns of $A$ are a basis for $\R^n$.

True. If the determinant is nonzero, the matrix is nonsingular, and its columns are linearly independent and span $\R^n$. Thus, the columns of the matrix are a basis for $\R^n$.

\end{enumerate}

\end{document}

\sffamily
