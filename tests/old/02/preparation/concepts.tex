\documentclass[12pt]{article}

\input{../../../../../fimacros.tex}

\setheadings{MTH 288 --- Linear Algebra --- Test 2 Concepts}

\begin{document}

\section{Questions}
For each of the following statements determine if it is true or false, and provide a brief justification for your answer.

It is \emph{guaranteed} that one or more of these questions will be in the test, in \emph{exactly the same form} as they appear below.


\begin{enumerate}

\item The set of solutions of the linear system:
\[
\begin{bmatrix}1&2&-1\\2&3&5\\-4&-2&3\end{bmatrix}
\begin{bmatrix}x_1\\x_2\\x_3\end{bmatrix}=
\begin{bmatrix}0\\0\\0\end{bmatrix}
\]
is a linear subspace of the euclidean vector space $\R^3$.

\item If $T:\R^3\to\R^4$ is a linear transformation, then $T$ is one-to-one.

\item If $T:\R^4\to\R^3$ is a linear transformation, then $T$ is onto.

\item If $T:\R^8\to\R^5$ is a linear transformation, then $T$ can be onto.

\item If $T:\R^8\to\R^5$ is a linear transformation, then $T$ can be one-to-one.

\item If $T:\R^4\to\R^4$ is a linear transformation, then $T$ can be invertible.

\item If a linear transformation $T:\R^3\to\R^3$ is one-to-one, then $T$ must also be onto.

\item If $A$ and $B$ are $n\times n$ matrices, then $(A+B)(A-B)=A^2-B^2$.

\item If $A$ is an $n\times n$ matrix and $A^3=0$, then $A=0$.

\item If $A$ and $B$ are $n\times n$ matrices, then $(AB)^T=B^TA^T$

\item If $A$ is an invertible $n\times n$ matrix, then the number of solutions to $A\mathbf{x}=\mathbf{b}$ is always greater than zero.

\item If the columns of an $n \times n$ matrix $A$ span $\R^n$, then $A$ is nonsingular.

\item If $A$ and $B$ are invertible, then $(AB)^{-1}=A^{-1}B^{-1}$.

\item The matrix
\[
\begin{bmatrix}2&3\\4&6\end{bmatrix}
\]
is invertible.

\item If $A$ is a $5\times 3$ matrix, then $\text{null}\,(A)$ forms a subspace of $\R^5$.

\item If $A$ is a $5\times 3$ matrix, then $\text{null}\,(A)$ forms a subspace of $\R^3$.

\item If $T:\R^5\to\R^8$ is a linear transformation, then $\text{range}\,(T)$ forms a subspace of $\R^8$.

\item The dimension of a subspace of $\R^5$ is one of the integers $1$, $2$, $3$ or $4$.

\item Let $\{\mathbf{u}_1,\mathbf{u}_2,\mathbf{u}_3\}$ be any set of $3$ vectors in $\R^5$. Then, it is always possible to add $2$ vectors to the set to form a basis of $\R^5$.

\item The only subspace of $\R^8$ that has dimension 8 is $\R^8$ itself.

\item $\R^3$ has infinitely many subspaces of dimension 2.

\item The rank of a matrix cannot exceed the number of rows of $A$.

\item If $A$ is not a square matrix, than the dimension of the row space of $A$ is different of the dimension of the column space of $A$.

\item If the system $A\mathbf{x}=\mathbf{b}$ has solutions, than $\mathbf{b}$ is in the row space of $A$.

\item If $A$ is an $n\times n$ matrix with all positive entries, then $\det(A)$ is positive.

\item If $A$ is an upper triangular square matrix, then $\det(A)\ne0$.

\item If $A$ and $B$ are $n\times n$ matrices, then $\det(A+B)=\det(A)+\det(B)$.

\item If $A$ is an $n\times n$ matrix and $c$ is a scalar, then $\det(cA)=c^n\det(A)$

\item If $a$ and $b$ are real numbers that are not both zero, then the matrix
\[
\begin{bmatrix}a&-b\\b&a\end{bmatrix}
\]
is nonsingular.

\item Let $A$ be an $n\times n$ matrix such that $\det(A)\ne 0$. Then the columns of $A$ are a basis for $\R^n$.


\end{enumerate}



\end{document}

\sffamily
