\documentclass[12pt]{article}

\input{../../../../../fimacros.tex}

\setheadings{MTH 288 --- Linear Algebra --- Test 2 Information}

\begin{document}

\sffamily

\section{Test Information}

\subsection{Time and Location}

\begin{center}
{\Large
Wednesday, April 13, in the Computer Lab (RT1501)
}
\end{center}

\subsection{Test Topics}

\begin{itemize}
\item Section 3.1: Linear Transformations
\item Section 3.2: Matrix Algebra
\item Section 3.3: Inverses
\item Section 4.1: Subspaces
\item Section 4.2: Basis and Dimension
\item Section 4.3: Row and Column Spaces
\item Section 5.1: The Determinant Function
\item Section 5.2: Properties of the Determinant
\item Section 5.3: Applications of the Determinant
\end{itemize}

Linear algebra is a discipline that is both \emph{computational} and \emph{conceptual}. To be successful in the test, it is essential to study both aspects of the subject and understand how they are related. The following tips will help you achieve this goal.

\section{Preparing For the Test}

\subsection{Preparation Strategies}

\begin{itemize}
\item Mathematics is a subject that can only be learned by \emph{doing}. Do as much practice problems as your time allows.
\item When studying, be an \emph{active reader}. Take notes as you read the text and the lecture notes, make a summary and/or a set of study cards.
\item Trying to simply memorize definitions, theorems and proofs is known to be a failing strategy to learn mathematics. Instead, it is important to develop an \emph{understanding} of how the several concepts fit together. Many people find it helpful to \emph{write} the concepts, since writing ``slows down'' our thought process to a reflection level that makes it easier to learn all the relationships.
\end{itemize}

\subsection{Test Policies}

\begin{itemize}
\item The test is with closed notes and books.
\item You can use the row operation software provided for the computer labs.
\item \emph{No other software is allowed.}
\item You are not allowed to visit any external web site during the test, or use the computer to communicate with anyone.
\item Calculators are allowed, but are not necessary. In fact, they are not recommended.
\item No scratch paper is allowed during the test. The test will contain enough space to contain all solutions.
\item Answers are to be given in exact form. That is, numbers must be expressed as rationals (fractions), not decimal approximations.
\item Solutions must contain the work that shows how answers were obtained. Solutions with no justification will receive no credit. (Unless where it is specifically said that a justification is not necessary.)
\item For the purpose of grading, all work must be written in the provided test. No printouts of Jupyter notebooks will be accepted. So, make sure you copy to the test writeup all the information you deem necessary for the solution of each problem.
\item Write the final answer of each problem in a manner that is consistent with the question asked. If, for example, a problem asks for the solution of a linear system a matrix will not be accepted as a valid answer.
\item \emph{The instructor will not answer questions during the test}. The only exception to this rule consists of readability issues caused by copying problems. Notice that \emph{being able to understand the statement of test problems is part of the skills being tested, and is each student's exclusive responsibility}.
\end{itemize}

\end{document}
















