\documentclass[12pt]{article}

\input{../../../../fimacros.tex}

\setheadings{MTH 288 --- Linear Algebra --- Test 2}

\begin{document}

Name and Student ID: \hrulefill


\textbf{Instructions.} All solutions must be justified, unless otherwise stated. Show all work leading to your answer in each problem. Solutions without appropriate work that supports it will receive no credit. All work must be written in the test. Do not attach computer printouts to the test. If not enough space is provided for an answer, continue it in the back of the page.

Please identify your final answer to each problem by surrounding it with a rectangle.


\clearpage


% Basis for subspace
\textbf{Problem 1.}(20 points.) 
Find, if possible, a basis for $\R^5$ that contains the vectors:

$$
\mathbf{u}_1=\left[\begin{matrix}2\\1\\3\\1\\4\end{matrix}\right]\quad
\mathbf{u}_2=\left[\begin{matrix}3\\-1\\0\\4\\1\end{matrix}\right]\quad
\mathbf{u}_3=\left[\begin{matrix}5\\1\\3\\5\\5\end{matrix}\right]\quad
$$
If it is not possible to find such a basis, explain why. Show work supporting your solution.

\clearpage

\textbf{Problem 2.}(20 points.)
Answer the following items for the matrix:
\[
A=
\begin{bmatrix}2&3&-1&7\\-3&0&-3&-6\\4&-5&9&3\\0&-1&1&-1\end{bmatrix}
\] 

\textbf{(a)} (10 points.) Find a basis for the row space of $A$, and determine the rank of $A$

\vfill
(Problem continues on next page).
\clearpage

\textbf{(b)} (10 points.) Find a basis for the column space of $A$.

\clearpage

\textbf{Problem 3.} (16 points.) Find, if possible, scalars $a$, $b$, $c$, $d$, $e$ and $f$ that make the following matrix identity true.
\[
\left[\begin{matrix}1 & 0 & 0\\a & 1 & 0\\b & c & 1\end{matrix}\right]
\left[\begin{matrix}1 & d & e\\0 & 1 & f\\0 & 0 & 1\end{matrix}\right]=
\left[\begin{matrix}1 & -4 & 6\\1 & -3 & 3\\-2 & 8 & -11\end{matrix}\right]
\]
If it is not possible to find such scalars, explain why. Show work supporting your solution. 

\clearpage

\textbf{Problem 4.} (16 points.) Find the determinants for the following matrices. Explain the method you used to find the determinant, showing all computations. Do not use Python for this problem.

\textbf{(a)} (8 points.)
$
\det\left[\begin{matrix}1 & -1 & 1\\2 & 3 & -4\\0 & -2 & 2\end{matrix}\right]
$

\vskip4in

\textbf{(b)} (8 points.)
$
\det\left[\begin{matrix}0 & 2 & 0 & 0\\1 & 4 & 0 & 3\\-1 & 3 & 2 & 1\\2 & 1 & 0 & 4\end{matrix}\right] 
$
\clearpage

\textbf{Problem 5.} (18 points.) Determine if each of the following statements is true or false, and provide a brief justification for your answer.

\textbf{(a)} (3 points.) If a linear transformation is defined by:
\[
T(\mathbf{x})=
\begin{bmatrix}
1&3&0&1&1&4\\0&2&-5&11&1&-1\\0&2&5&-3&-4&3
\end{bmatrix}\mathbf{x},
\]
then the kernel of $T$ is a subspace of $\R^3$.

Circle one:\ \ \ \ \  True\ \ \ \ \ False

Justification:

\vskip1.5in


\textbf{(b)} (3 points.) A linear transformation $T:\R^4\to\R^4$ that is one-to-one is necessarily onto.

Circle one:\ \ \ \ \  True\ \ \ \ \ False

Justification:

\vskip1.5in


\textbf{(c)} (3 points.) If $A$ and $B$ are nonsingular, then $\det(A+B)\ne0$. 


Circle one:\ \ \ \ \  True\ \ \ \ \ False

Justification:

\clearpage

\textbf{(d)} (3 points.) Let $A$ and $B$ be the matrices:
\[
A=\begin{bmatrix}a&b&c\\d&e&f\\g&h&i\end{bmatrix};\quad
B=\begin{bmatrix}d&2e&f\\a&2b&c\\g&2h&i\end{bmatrix}
\]
Then, $\det(B)=-2\det(A)$

Circle one:\ \ \ \ \  True\ \ \ \ \ False

Justification:

\vskip1.5in


\textbf{(e)} (3 points.) If $\mathbf{u}_1$, $\mathbf{u}_2$ and $\mathbf{u}_3$ are linearly independent vectors in $\R^4$, then the subspace spanned by $\{\mathbf{u}_1,\mathbf{u}_2,\mathbf{u}_3\}$ has dimension 3.

Circle one:\ \ \ \ \  True\ \ \ \ \ False

Justification:

\vskip1.5in

\textbf{(e)} (3 points.) A $2\times 2$ matrix that has three positive entries and one negative entry can be singular.

Circle one:\ \ \ \ \  True\ \ \ \ \ False

Justification:

\clearpage

\textbf{Problem 6.} (10 points.) Find all real numbers $a$ for which the matrix below is singular:
\[
\left[\begin{matrix}a & 1 & 0\\2 & a & 3\\0 & 0 & a\end{matrix}\right]
\]

\end{document}