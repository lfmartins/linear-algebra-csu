\documentclass[12pt]{article}

\input{../../../../fimacros.tex}

\setheadings{MTH288 --- Practice with Bases and Matrix Inverses}

\begin{document}
\textbf{Exercise 1.} Let's determine if the following vectors form a basis of $\R^3$:
\[
\mathbf{v}_1=\begin{bmatrix*}[r] 2  \\ 2 \\ 1 \end{bmatrix*}\quad
\mathbf{v}_2=\begin{bmatrix*}[r] -3  \\ 2 \\ -3 \end{bmatrix*}\quad
\mathbf{v}_3=\begin{bmatrix*}[r] 0 \\ 2 \\ 2 \end{bmatrix*}
\]

\textbf{(a)} Write, in matrix form, the system of equations that is equivalent to finding a representation of an arbitrary vector as a linear combination of $\mathbf{v}_1,\mathbf{v}_2,\mathbf{v}_3$:
\[
\begin{bmatrix} x \\ y \\ z\end{bmatrix} = c_1\mathbf{v}_1+c_2\mathbf{v}_2+c_3\mathbf{v}_3
\]

\vskip2in

\textbf{(b)} Solve the system you found in the previous item using Gaussian Elimination. Write in the space below the RREF equivalent, and the system corresponding to the RREF.

\vskip3in

\textbf{(c)} What is your conclusion? Is the given set of vectors a basis? Explain why or why not.

\clearpage

\textbf{Exercise 2.} Let's determine if the following vectors form a basis of $\R^4$:
\[
\mathbf{v}_1=\begin{bmatrix*}[r] 1  \\ 4 \\ 0 \\ 2 \end{bmatrix*}\quad
\mathbf{v}_2=\begin{bmatrix*}[r] 1  \\ 1 \\ 2  \\ 2\end{bmatrix*}\quad
\mathbf{v}_3=\begin{bmatrix*}[r] 0 \\ 0 \\ 0 \\ 3 \end{bmatrix*}\quad
\mathbf{v}_4=\begin{bmatrix*}[r] 1 \\ 1 \\ 1 \\ 1 \end{bmatrix*}
\]

\textbf{(a)} Write, in matrix form, the system of equations that is equivalent to finding a representation of an arbitrary vector as a linear combination of $\mathbf{v}_1,\mathbf{v}_2,\mathbf{v}_3$:
\[
\begin{bmatrix} r \\ s \\ t \\ u\end{bmatrix} = c_1\mathbf{v}_1+c_2\mathbf{v}_2+c_3\mathbf{v}_3+c_4\mathbf{v}_4
\]

\vskip2in

\textbf{(b)} Solve the system you found in the previous item using Gaussian Elimination. Write in the space below the RREF equivalent, and the system corresponding to the RREF.

\vskip2.8in

\textbf{(c)} What is your conclusion? Is the given set of vectors a basis? Explain why or why not.

\clearpage

\textbf{Exercise 3.} Let's determine if the following vectors form a basis of $\R^4$:
\[
\mathbf{v}_1=\begin{bmatrix*}[r] -1  \\ 2 \\ 1 \\ 2 \end{bmatrix*}\quad
\mathbf{v}_2=\begin{bmatrix*}[r] -1  \\ 1 \\ 2  \\ 2\end{bmatrix*}\quad
\mathbf{v}_3=\begin{bmatrix*}[r] -2 \\ 3 \\ 3 \\ 4 \end{bmatrix*}\quad
\mathbf{v}_4=\begin{bmatrix*}[r] 1 \\ 2 \\ 3 \\ 3 \end{bmatrix*}
\]

\textbf{(a)} Write, in matrix form, the system of equations that is equivalent to finding a representation of an arbitrary vector as a linear combination of $\mathbf{v}_1,\mathbf{v}_2,\mathbf{v}_3$:
\[
\begin{bmatrix} r \\ s \\ t \\ u\end{bmatrix} = c_1\mathbf{v}_1+c_2\mathbf{v}_2+c_3\mathbf{v}_3+c_4\mathbf{v}_4
\]

\vskip2in

\textbf{(b)} Solve the system you found in the previous item using Gaussian Elimination. Write in the space below the RREF equivalent, and the system corresponding to the RREF.

\vskip2.8in

\textbf{(c)} What is your conclusion? Is the given set of vectors a basis? Explain why or why not.

\clearpage

\textbf{Exercise 4.} Let $A$ be the matrix:
\[
A=\left[\begin{matrix*}[r]1 & 1 & 0 & 2\\2 & 0 & 0 & -2\\-3 & 1 & 1 & 4\\2 & 3 & 4 & 1\end{matrix*}\right]
\]

Find, if possible, the inverse of the matrix $A$ by augmenting it with the identity matrix and using Gaussian Elimination to find its RREF. In the space below, write the augmented matrix, the row operations used, the RREF of the matrix and, finally, the inverse of the matrix $A$. If the matrix is not invertible, provide a justification for your conclusion.

\clearpage

\textbf{Exercise 5.} Let $A$ be the matrix:
\[
A=\left[\begin{matrix*}[r]2 & 2 & 1 & 3\\-1 & 3 & 4 & 0\\-7 & -3 & 1 & -9\\1 & 13 & 14 & 6\end{matrix*}\right]
\]

Find, if possible, the inverse of the matrix $A$ by augmenting it with the identity matrix and using Gaussian Elimination to find its RREF. In the space below, write the augmented matrix, the row operations used, the RREF of the matrix and, finally, the inverse of the matrix $A$. If the matrix is not invertible, provide a justification for your conclusion.

\clearpage

\textbf{Exercise 6.} Let $A$ be the matrix:
\[
A=\left[\begin{matrix*}[r]a & b \\c & d\end{matrix*}\right]
\]

Find, if possible, the inverse of the matrix $A$ by augmenting it with the identity matrix and using Gaussian Elimination to find its RREF. In the space below, write the augmented matrix, the row operations used, the RREF of the matrix and, finally, the inverse of the matrix $A$. If the matrix is not invertible, provide a justification for your conclusion.

\emph{Note}: For this problem you can't use the computer, so do the calculations by hand.

\clearpage


\end{document}




























