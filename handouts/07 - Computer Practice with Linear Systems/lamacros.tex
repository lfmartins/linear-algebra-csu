\RequirePackage{amsmath}
\RequirePackage{amsfonts}
\RequirePackage{amssymb}
\RequirePackage{amsthm}
\RequirePackage{mathtools}
\RequirePackage{graphicx}
\RequirePackage{mathabx}
\RequirePackage{varioref}
\RequirePackage{cancel}
\RequirePackage{MnSymbol,wasysym}
\RequirePackage{pdfpages}

\RequirePackage{listings}
\lstset{language=Python, 
   basicstyle=\ttfamily\small, 
   showstringspaces=false,
   tabsize=3,
   frameround=tttt,
   captionpos=b}

\lstnewenvironment{lstfancy}[2]
{\lstset{float=th,caption=#1,label=#2,numbers=left,numberstyle=\tiny,%
frame=tb,escapeinside={(*}{*)}}}
{}

\RequirePackage[colorlinks=true,
	linkcolor=blue, urlcolor=blue,
	pdfauthor={L. Felipe Martins},
	pdfsubject={Computational number theory},
	pdfkeywords={number theory, computations, cryptography},
	pdfstartview=FitH,
	pdfview=FitH]{hyperref}

\RequirePackage{fancyhdr}

\RequirePackage{titlesec}
\titleformat{\section}{\large\bf\sffamily}{\thesection}{1em}{}
\titleformat{\subsection}{\bf\sffamily}{\thesubsection}{1em}{}

\newenvironment{remarkbox}%
{\begin{Sbox}\begin{minipage}{6in}}%
{\end{minipage}\end{Sbox}\begin{center}\shadowbox{\TheSbox}\end{center}}%



%----------------------------------------------------------
%\swapnumbers
\theoremstyle{plain}
\newtheorem{theorem}{Theorem}[section]
\newtheorem{proposition}[theorem]{Proposition}
\newtheorem{lemma}[theorem]{Lemma}
\newtheorem{corollary}[theorem]{Corollary}
\newtheorem{conjecture}[theorem]{Conjecture}
\newtheorem*{conjecture*}{Conjecture}
\newtheorem{criterion}[theorem]{Criterion}
\newtheorem{principle}[theorem]{Principle}
\newtheorem{property}[theorem]{Property}

\theoremstyle{definition}
\newtheorem{definition}[theorem]{Definition}
\newtheorem{condition}[theorem]{Condition}
\newtheorem{example}[theorem]{Example}
\newtheorem{exercise}{}[section]
\newtheorem*{solution}{\emph{\textbf{Solution}}}

\theoremstyle{remark}
\newtheorem{remark}[theorem]{Remark}
\newtheorem{note}[theorem]{Note}
\newtheorem{notation}[theorem]{Notation}
\newtheorem{claim}[theorem]{Claim}
%----------------------------------------------------------

%My definitions

%Number sets
\newcommand\Z{\ensuremath{\mathbb{Z}}}
\newcommand\N{\ensuremath{\mathbb{N}}}
\newcommand\R{\ensuremath{\mathbb{R}}}
\newcommand\Q{\ensuremath{\mathbb{Q}}}
\newcommand\C{\ensuremath{\mathbb{C}}}

% Floor and ceil functions
\newcommand{\floorf}[1]{\ensuremath{\left\lfloor{#1}\right\rfloor}}
\newcommand{\ceilf}[1]{\ensuremath{\left\lceil{#1}\right\rceil}}

% "Divides" and "Not divides"
\newcommand{\dv}{\ensuremath{\mathbf{\divides}}}
\newcommand{\ndv}{\ensuremath{\notdivides}}

%Shortcut to underline and cancel
\newcommand{\un}[1]{\underline{#1}}
\newcommand{\cn}[1]{\cancel{#1}}

% Sets
\newcommand{\setof}[2]{\ensuremath{\left\{ #1 \; | \; #2 \right\}}}
\newcommand{\setofc}[2]{\ensuremath{\left\{ #1 : #2 \right\}}}
\newcommand{\setenum}[1]{\left\{#1\right\}}

% Prime factorizations
%\newcommand{\primefactorization}[3]{\ensuremath{#1_1^{#2_1}#1_2^{#2_2}\ldots#1_{#3}^{#2_{#3}}}}
\newcommand{\primefactorization}[3]{#1_1^{#2_1}#1_2^{#2_2}\ldots#1_{#3}^{#2_{#3}}}
\newcommand{\primefactorizationprod}[4]{\ensuremath{\prod_{#4=1}^{#3}#1_{#4}^{#2_{#4}}}}

% Modular identities
\newcommand{\eqmod}[3]{\ensuremath{#1 \equiv #2 \pmod{#3}}}

% End of proof
\newcommand{\proofend}{\hfill  {\Large\smiley}}%


%Legendre symbols
%\newcommand{\legendre}[2]{ {#1}\abovewithdelims () {#2}}
\newcommand{\legendre}[2]{\left(\frac{#1}{#2}\right)}

% Operators "div" and "divmod" 
%\DeclareMathOperator{\bdiv}{div}
\newcommand{\bdiv}{\;\mathrm{div}\;}
\DeclareMathOperator{\round}{round}
%\DeclareMathOperator{\divmod}{divmod}
%\DeclareMathOperator{\bdivp}{divp}
%\DeclareMathOperator{\bmodp}{modp}
%\DeclareMathOperator{\divmodp}{divmodp}
\DeclareMathOperator{\ord}{ord}
\DeclareMathOperator{\dlog}{dlog}

%Equation reference formats
\labelformat{equation}{(#1)}

%Macro to introduce license notice
%\newcommand{\license}%
%{{\scriptsize Work licensed under a Creative Commons License available at}\\%
%{\scriptsize\url{http://creativecommons.org/licenses/by-nc-sa/3.0/us/}}\\}

%--------------------- Formatting for number theory modules ------------------------
% Set headings
\newcommand{\setheadings}[1]{%
\fancyhead[L]{{\large \textsf{#1}}}%
\fancyhead[R]{\thepage}%
\fancyfoot[L]{%
{\scriptsize L. Felipe Martins (\href{mailto:l.martins@csuohio.edu}{l.martins@csuohio.edu})\\%
%Licensed under a Creative Commons License, visit %\url{http://creativecommons.org/licenses/by-nc-sa/3.0/} to view a copy of the license.%
}}}%

\newcommand{\setheadingsa}[2]{%
\fancyhead[L]{{\large \textsf{#1}}}%
\fancyhead[R]{\thepage}%
\fancyfoot[L]{\scriptsize{#2}}%
}%

\renewcommand{\footrulewidth}{0.4pt}%

% Change margins
\setlength{\headheight}{29pt}
\pagestyle{fancy}
\fancyhead{}
\fancyfoot{}
\usepackage[left=2.5cm,right=2cm,bottom=2cm]{geometry}
\fancyheadoffset[L]{0cm}

\numberwithin{equation}{section}

\setlength{\parindent}{0pt}
\setlength{\parskip}{0.5em}

