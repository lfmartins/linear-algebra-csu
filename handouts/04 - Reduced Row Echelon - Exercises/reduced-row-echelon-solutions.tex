\documentclass[12pt]{article}

\usepackage{mathtools}

\input{../../../../fimacros.tex}

\setheadings{MTH288 --- Reduced Row Echelon Exercises}

\begin{document}
Each of the items below is the augmented matrix of a linear system of equations. Your task is, for each matrix:
\begin{enumerate}
\item Identify if the matrix is in reduced row echelon form.
\item If the matrix is not in RREF, explain why.
\item If the matrix is in RREF, write the solution of the system using set notation.
\end{enumerate}

\begin{enumerate}
\item
$\begin{bmatrix*}[r]
1 & 0 & 0 &  2 & -3\\
0 & 1 & 0 & -2 & 1\\
0 & 0 & 1 &  1 & 0\\
\end{bmatrix*}$

\emph{Solution}:
We first rewrite the augmented matrix as a system. The augmented matrix has 5 columns, so there are 4 variables, since the last column corresponds to the right-hand side of the system. We choose the names $x$, $y$, $z$, $t$ for the variables:
\begin{alignat*}{8}
x &{}\quad{}&   &{}\quad{}&   &{}+{}& 2t &{}={}& -3\\
  &{}{}& y &{}{}&   &{}-{}& 2t &{}={}&  1\\
  &{}{}&   &{}{}& z &{}+{}&  t &{}={}&  0 
\end{alignat*}
There is a standard way to interpret the structure of the system for a RREF:
\begin{enumerate}
\item Variables that correspond to columns with pivots are called \emph{pivot variables}. In our example, these are $x$, $y$ and $z$.
\item Variables that correspond to columns that do not have pivots are called \emph{free variables}. In this example, the only free variable is $t$. 
\end{enumerate}
The RREF yields an easy way to express the pivot variables in terms of the free variables:
\begin{alignat*}{4}
x &{}={}& -3 &{}-{}& 2t\\
y &{}={}&  1 &{}+{}& 2t\\
z &{}={}&    &{}-{}& t
\end{alignat*}
The above is one way to express the solutions of the system. We can freely choose the value of $t$, and then compute $x$, $y$ and $z$. We now want to express the solution as a vector:
\[
\begin{bmatrix}x\\y\\z\\t\end{bmatrix}
\]
We choose a parameter to represent the free variable $t$:
\[
t=\lambda_1
\]
We can then represent the whole solution (including the free variable $t$) in terms of the parameters $\lambda_1$
\begin{alignat*}{4}
x &{}={}& -3 &{}-{}& 2\lambda_1\\
y &{}={}&  1 &{}+{}& 2\lambda_1\\
z &{}={}&    &{}-{}&  \lambda_1\\
t &{}={}&    &{}+{}&  \lambda_1
\end{alignat*}
The vector form of the solution can be directly read from this representation:
\[
\begin{bmatrix}x\\y\\z\\t\end{bmatrix}=
\begin{bmatrix*}[r]3\\1\\0\\0\end{bmatrix*}+
\lambda_1
\begin{bmatrix*}[r]-2\\2\\-1\\1\end{bmatrix*}
\]
The \emph{solution set} is the set of all vectors that are solutions of the system, which, in this case, is:
\[
\left\{
\begin{bmatrix*}[r]3\\1\\0\\0\end{bmatrix*}+
\lambda_1
\begin{bmatrix*}[r]-2\\2\\-1\\1\end{bmatrix*}
\;:\,\lambda_1\in\R\right\}
\]

We read the above notation as follows:

``\emph{The set of all vectors of the form
\[
\begin{bmatrix*}[r]3\\1\\0\\0\end{bmatrix*}+
\lambda_1
\begin{bmatrix*}[r]-2\\2\\-1\\1\end{bmatrix*}
\]
where $\lambda_1$ is a real number}''
\proofend


\bigskip
\item
$\begin{bmatrix*}[r]
1 & 2 & 0 &  3 & 5\\
0 & 1 & 0 & -2 & 1\\
0 & 0 & 1 & -4 & 3\\
0 & 0 & 0 &  3  & 2\\
\end{bmatrix*}$

\emph{Solution}: This matrix is not in RREF because the leading element of the last row is 3, not 1.
\proofend

\bigskip
\item
$\begin{bmatrix*}[r]
1 & 0 & 2 & 0 & 3 & 0 & -5\\
0 & 0 & 0 & 1 & 4 & 0 & 7\\
0 & 0 & 0 & 0 & 0 & 1 & 12\\
0 & 0 & 0 & 0 & 0 & 0 & 0\\
0 & 0 & 0 & 0 & 0 & 0 & 0\\
\end{bmatrix*}$

\emph{Solution}: The system that corresponds to this matrix, using the variables $a$, $b$,\dots, $f$ is:
\begin{alignat*}{12}
 a  &{}{}& &{}+{}& 2c&{}{}&   &{}+{}& 3e &{}{}&    &{}={}& -5\\
    &{}{}& &{}{}&    &{}{}& d &{}+{}& 4e &{}{}&    &{}={}&  7\\
    &{}{}& &{}{}&    &{}{}&   &{}{}&     &{}{}&  f &{}={}&  12\\
    &{}{}& &{}{}&    &{}{}&   &{}{}&     &{}{}&  0 &{}={}&  0\\
    &{}{}& &{}{}&    &{}{}&   &{}{}&     &{}{}&  0 &{}={}&  0
\end{alignat*}
Notice that this system has a peculiarity: the second column of the RREF, corresponding to the variable $b$, is all zero. This means that the system does not impose any conditions on the variable $b$, so that it can be chosen arbitrarily. Also notice that the two last equations, $0=0$, are redundant (they don't impose any new conditions). 

The variables can be identified as follows:
\begin{itemize}
\item Pivot variables: $a$, $d$ and $f$
\item Free variables: $b$, $c$ and $e$.
\end{itemize}
Expressing the pivot variables in terms of the free variables we get:
\begin{alignat*}{6}
a &{}={}& -5 &{}-{}& 2c &{}-{}& 3e\\
d &{}={}&  7 &{}{}&     &{}-{}& 4e\\
f &{}={}& 12 &{}{}&     &{}{}&
\end{alignat*} 
Let's now choose one parametric variable for each of the free variables:
\begin{align*}
b &= \lambda_1 \text{ (don't forget this!)}\\
c &= \lambda_2\\
e &= \lambda_3
\end{align*}
Writing the solution in terms of the parametric variables:
\begin{alignat*}{6}
a &{}={}& -5 &{}\quad{}&           &{}-{}& 2\lambda_2 &{}-{}& 3\lambda_3\\
b &{}={}&    &{}{}& \lambda_1 &{}{}&             &{}{}&\\
c &{}={}&    &{}{}&           &{}{}&  \lambda_2  &{}{}&\\
d &{}={}&  7 &{}{}&           &{}{}&             &{}-{}& 4\lambda_3\\
e &{}={}&    &{}{}&           &{}{}&             &{}{}&   \lambda_3\\
f &{}={}& 12
\end{alignat*} 
In vector form:
\[
\begin{bmatrix}a\\b\\c\\d\\e\\f\end{bmatrix}=
\begin{bmatrix*}[r]-5\\ 0\\ 0\\ 7\\ 0\\ 12\end{bmatrix*}
+\lambda_1\begin{bmatrix*}[r] 0\\ 1\\ 0\\  0\\  0\\ 0 \end{bmatrix*}
+\lambda_2\begin{bmatrix*}[r]-2\\ 0\\ 1\\  0\\  0\\ 0 \end{bmatrix*}
+\lambda_3\begin{bmatrix*}[r]-3\\ 0\\ 0\\ -4\\  1\\ 0\\ \end{bmatrix*}
\]
In set notation:
\[
\left\{
\begin{bmatrix*}[r]-5\\ 0\\ 0\\ 7\\ 0\\ 12\end{bmatrix*}
+\lambda_1\begin{bmatrix*}[r] 0\\ 1\\ 0\\  0\\  0\\ 0 \end{bmatrix*}
+\lambda_2\begin{bmatrix*}[r]-2\\ 0\\ 1\\  0\\  0\\ 0 \end{bmatrix*}
+\lambda_3\begin{bmatrix*}[r]-3\\ 0\\ 0\\ -4\\  1\\ 0\\ \end{bmatrix*}
\;:\;
\lambda_1,\lambda_2,\lambda_3\in\R\right\}
\]
\endproof

\bigskip
\item
$\begin{bmatrix*}[r]
1 & 2 & 3 & 4 & 5\\
\end{bmatrix*}$

\emph{Solution}:
The system is, using the variables $x$, $y$, $u$, $v$:
\[
x+2y+3u+4v=5
\]
\begin{itemize}
\item Pivot variable: $x$
\item Free variables: $y$, $u$, $v$
\end{itemize}
Expressing the pivot variable in terms of the free variables:
\[
x = 5-2y-3u-4v
\]
Choose a parametric variable for each free variable:
\begin{align*}
y &= \lambda_1\\
u &= \lambda_2\\
v &= \lambda_3
\end{align*}
Expressing the solution in terms of the parametric variables:
\begin{alignat*}{8}
x &{}={}& 5 &{}-{}& 2\lambda_1 &{}-{}& 3\lambda_2 &{}-{}& 4\lambda_3\\
y &{}={}&   &{}{}&   \lambda_1 &{}{}&             &{}{}&\\
u &{}={}&   &{}{}&             &{}{}&   \lambda_2 &{}{}&\\
v &{}={}&   &{}{}&             &{}{}&             &{}{}& \lambda_3
\end{alignat*}
Vector form:
\[
\begin{bmatrix}x\\y\\u\\v\end{bmatrix}=
\begin{bmatrix*}[r] 5\\ 0\\ 0\\ 0\end{bmatrix*}
+\lambda_1\begin{bmatrix*}[r] -2\\ 1\\ 0\\ 0\end{bmatrix*}
+\lambda_2\begin{bmatrix*}[r] -3\\ 0\\ 1\\ 0\end{bmatrix*}
+\lambda_3\begin{bmatrix*}[r] -4\\ 0\\ 0\\ 1\end{bmatrix*}
\]
Set notation:
\[
\left\{
\begin{bmatrix*}[r] 5\\ 0\\ 0\\ 0\end{bmatrix*}
+\lambda_1\begin{bmatrix*}[r] -2\\ 1\\ 0\\ 0\end{bmatrix*}
+\lambda_2\begin{bmatrix*}[r] -3\\ 0\\ 1\\ 0\end{bmatrix*}
+\lambda_3\begin{bmatrix*}[r] -4\\ 0\\ 0\\ 1\end{bmatrix*}
\;:\;
\lambda_1,\lambda_2,\lambda_3\in\R\right\}
\]

\bigskip
\item
$\begin{bmatrix*}[r]
1 & 2 & 3 & 0 & 2  & 0 & 5 & 2\\
0 & 0 & 0 & 1 & -2 & 0 & 3 & -8\\
0 & 0 & 0 & 0 & 0  & 1 & 4 & 10\\
\end{bmatrix*}$

\emph{Solution}:

\emph{Note}: This solution describes a streamlined solution method, which is equivalent to what was done for the examples above. \emph{Do not use this method unless you feel comfortable about what you are doing!}

First notice that we have 7 columns that correspond to the variables:
\begin{itemize}
\item Pivot columns: 1, 4 and 6.
\item Non-pivot columns: 2, 3, 5 and 7.
\end{itemize}
Lets call the variables $x_1$, $x_2$,\dots $x_7$. Then, the free variables are $x_2$, $x_3$, $x_5$ and $x_7$, and we need four parametric variables:
\begin{align*}
x_2 &= \lambda_1\\
x_3 &= \lambda_2\\
x_5 &= \lambda_3\\
x_7 &=\lambda_4
\end{align*}
The vector representation of the solution will have the following structure:
\[
\begin{bmatrix}x_1\\x_2\\x_3\\x_4\\x_5\\x_6\\x_7\end{bmatrix}
=
\begin{bmatrix*}[r]           \quad \\ \\ \\ \\ \\ \\ \\ \end{bmatrix*}
+\lambda_1\begin{bmatrix*}[r] \quad \\ \\ \\ \\ \\ \\ \\ \end{bmatrix*}
+\lambda_2\begin{bmatrix*}[r] \quad \\ \\ \\ \\ \\ \\ \\ \end{bmatrix*}
+\lambda_3\begin{bmatrix*}[r] \quad \\ \\ \\ \\ \\ \\ \\ \end{bmatrix*}
+\lambda_4\begin{bmatrix*}[r] \quad \\ \\ \\ \\ \\ \\ \\ \end{bmatrix*}
\]
We now have to fill the entries. We can start with the rows corresponding to the \emph{free variables}:
\[
\begin{bmatrix}x_1\\x_2\\x_3\\x_4\\x_5\\x_6\\x_7\end{bmatrix}
=
\begin{bmatrix*}[r]           \quad \\ 0\\ 0\\ \\ 0\\ \\ 0\end{bmatrix*}
+\lambda_1\begin{bmatrix*}[r] \quad \\ 1\\ 0\\ \\ 0\\ \\ 0\end{bmatrix*}
+\lambda_2\begin{bmatrix*}[r] \quad \\ 0\\ 1\\ \\ 0\\ \\ 0\end{bmatrix*}
+\lambda_3\begin{bmatrix*}[r] \quad \\ 0\\ 0\\ \\ 1\\ \\ 0\end{bmatrix*}
+\lambda_4\begin{bmatrix*}[r] \quad \\ 0\\ 0\\ \\ 0\\ \\ 1\end{bmatrix*}
\]
(Notice that, if we look only at the rows that have the free variables, we have a pattern that is reminiscent of the identity matrix!)
We can now fill the rows for the other variables. For example, from the first row we get the equation for $x_1$:
\[
x_1=2-2x_2-3x_3-2x_5-5x_7=2-2\lambda_1-3\lambda_2-2\lambda_3-5\lambda_4
\]
so that we get:
\[
\begin{bmatrix}x_1\\x_2\\x_3\\x_4\\x_5\\x_6\\x_7\end{bmatrix}
=
\begin{bmatrix*}[r]            2 \\ 0\\ 0\\ \\ 0\\ \\ 0\end{bmatrix*}
+\lambda_1\begin{bmatrix*}[r] -2 \\ 1\\ 0\\ \\ 0\\ \\ 0\end{bmatrix*}
+\lambda_2\begin{bmatrix*}[r] -3 \\ 0\\ 1\\ \\ 0\\ \\ 0\end{bmatrix*}
+\lambda_3\begin{bmatrix*}[r] -2 \\ 0\\ 0\\ \\ 1\\ \\ 0\end{bmatrix*}
+\lambda_4\begin{bmatrix*}[r] -5 \\ 0\\ 0\\ \\ 0\\ \\ 1\end{bmatrix*}
\]
Filling the remaining entries, we get:
\[
\begin{bmatrix}x_1\\x_2\\x_3\\x_4\\x_5\\x_6\\x_7\end{bmatrix}
=
\begin{bmatrix*}[r]            2 \\ 0\\ 0\\ -8\\ 0\\ 10\\ 0\end{bmatrix*}
+\lambda_1\begin{bmatrix*}[r] -2 \\ 1\\ 0\\  0\\ 0\\  0\\ 0\end{bmatrix*}
+\lambda_2\begin{bmatrix*}[r] -3 \\ 0\\ 1\\  0\\ 0\\  0\\ 0\end{bmatrix*}
+\lambda_3\begin{bmatrix*}[r] -2 \\ 0\\ 0\\  2\\ 1\\  0\\ 0\end{bmatrix*}
+\lambda_4\begin{bmatrix*}[r] -5 \\ 0\\ 0\\ -3\\ 0\\ -4\\ 1\end{bmatrix*}
\]
\emph{A word of advice}: Be careful with the signs, and where the zero entries should be. After you are finished, as a check, you can work in reverse from the vector representation to see if you get the right RREF.

Finally, in terms of set notation:
\[
\left\{
\begin{bmatrix*}[r]            2 \\ 0\\ 0\\ -8\\ 0\\ 10\\ 0\end{bmatrix*}
+\lambda_1\begin{bmatrix*}[r] -2 \\ 1\\ 0\\  0\\ 0\\  0\\ 0\end{bmatrix*}
+\lambda_2\begin{bmatrix*}[r] -3 \\ 0\\ 1\\  0\\ 0\\  0\\ 0\end{bmatrix*}
+\lambda_3\begin{bmatrix*}[r] -2 \\ 0\\ 0\\  2\\ 1\\  0\\ 0\end{bmatrix*}
+\lambda_4\begin{bmatrix*}[r] -5 \\ 0\\ 0\\ -3\\ 0\\ -4\\ 1\end{bmatrix*}
\;:\;\lambda_1,\lambda_2,\lambda_3,\lambda_4\in\R\right\}
\]
\proofend

\bigskip
\item
$\begin{bmatrix*}[r]
1 & 0 &  2 & 0 & 0 & 0 &  3 &  4 & 9\\
0 & 1 & -2 & 0 & 0 & 0 & -4 &  0 & 0\\
0 & 0 &  0 & 1 & 0 & 0 &  0 & -1 & 7\\
0 & 0 &  0 & 0 & 1 & 3 &  2 &  2 & 8\\
\end{bmatrix*}$

\emph{Solution}: Using $x_1$, $x_2$, \dots, $x_8$ as variables:
\begin{itemize}
\item Pivot variables: $x_1$, $x_2$, $x_4$, $x_5$.
\item Free variables: $x_3$, $x_6$, $x_7$, $x_8$.
\end{itemize}
Parametric variables:
\begin{align*}
x_3 &= \lambda_1\\
x_6 &= \lambda_2\\
x_7 &= \lambda_3\\
x_8 &= \lambda_4
\end{align*}
Filling the entries of the vector representation we get:
\[
\begin{bmatrix}     x_1\\ x_2\\ x_3\\ x_4\\ x_5\\ x_6\\ x_7\\ x_8\\\end{bmatrix}=
\begin{bmatrix*}[r]   9\\   0\\   0\\   7\\   8\\   0\\   0\\   0\\\end{bmatrix*}+
\lambda_1
\begin{bmatrix*}[r]  -2\\   2\\   1\\   0\\   0\\   0\\   0\\   0\\\end{bmatrix*}+ 
\lambda_2
\begin{bmatrix*}[r]   0\\   0\\   0\\   0\\  -3\\   1\\   0\\   0\\\end{bmatrix*}+ 
\lambda_3
\begin{bmatrix*}[r]  -3\\   4\\   0\\   0\\  -2\\   0\\   1\\   0\\\end{bmatrix*}+ 
\lambda_4
\begin{bmatrix*}[r]  -4\\   0\\   0\\   1\\  -2\\   0\\   0\\   1\\\end{bmatrix*} 
\]
\proofend

\bigskip
\item
$\begin{bmatrix*}[r]
1 & 0 &  2 & 0 & 1 & 4\\
0 & 1 & -4 & 0 & 0 & 3\\
0 & 0 &  0 & 1 & 2 & 5\\
0 & 0 &  0 & 0 & 0 & 1\\
\end{bmatrix*}$

\emph{Solution}:
When we write this as a linear system, the last equation will be:
\[
0=1
\]
Since this equation is always false, no matter what are the values of the variables, the system has no solutions. We call such a system \emph{inconsistent}. The solution set is the \emph{empty set}:
\[
\text{Solution set}=\varnothing
\]
\proofend

%\item
%$\begin{bmatrix*}[r]
%\end{bmatrix*}$

%\item
%$\begin{bmatrix*}[r]
%\end{bmatrix*}$

%\item
%$\begin{bmatrix*}[r]
%\end{bmatrix*}$

\end{enumerate}

\end{document}




































