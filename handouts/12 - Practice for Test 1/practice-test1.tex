\documentclass[12pt]{article}

\input{../../../../fimacros.tex}

\setheadings{MTH288 --- Practice for Test 1}

\begin{document}
\textbf{Exercise 1.} Find the solution set for the following systems. To receive full credit, your must include the following:
\begin{itemize}
\item The augmented matrix corresponding to the system.
\item The elementary row operations used to find the RREF equivalent to the augmented matrix. Write the row operations in the order they were used and the final RREF matrix. It is not necessary to write the intermediate matrix.
\item The solution set, a set of vectors represented as a set of vectors represented in parametric form.
\end{itemize} 

\begin{enumerate}
\item \begin{alignat*}{15}
&{}{}&&{}{}&x_{3} &{}-{}& x_{4} &{}+{}& x_{5} &{}+{}& 5 x_{6} &{}={}&0\\ 
x_{1} &{}-{}& 2 x_{2} &{}+{}& x_{3} &{}-{}& x_{4}&{}{}& &{}+{}& x_{6} &{}={}&-8\\ 
- 3 x_{1} &{}+{}& 6 x_{2} &{}+{}& 2 x_{3} &{}-{}& 2 x_{4} &{}-{}& x_{5} &{}+{}& 10 x_{6} &{}={}&6\\ 
- 2 x_{1} &{}+{}& 4 x_{2}&{}{}&&{}{}& &{}-{}& 3 x_{5} &{}-{}& 2 x_{6} &{}={}&1 
\end{alignat*}

\item \begin{alignat*}{15}
x_{1} &{}-{}& x_{2} &{}-{}& 2 x_{3} &{}+{}& 2 x_{4} &{}+{}& x_{5} &{}-{}& 3 x_{6} &{}={}&-1\\ 
x_{1} &{}+{}& 2 x_{2} &{}+{}& 2 x_{3} &{}+{}& 2 x_{4} &{}-{}& 28 x_{5} &{}-{}& x_{6} &{}={}&7\\ 
- 3 x_{1} &{}-{}& x_{2} &{}+{}& 2 x_{3} &{}-{}& 3 x_{4} &{}+{}& 14 x_{5} &{}-{}& x_{6} &{}={}&17\\ 
- 2 x_{1} &{}-{}& x_{2} &{}-{}& 2 x_{3} &{}-{}& x_{4} &{}+{}& 22 x_{5} &{}+{}& x_{6} &{}={}&-1\\ 
&{}{}&- x_{2}&{}{}& &{}-{}& 3 x_{4} &{}+{}& 18 x_{5} &{}+{}& x_{6} &{}={}&-5 
\end{alignat*}

\item \begin{alignat*}{11}
- 2 x_{1} &{}-{}& 2 x_{2} &{}-{}& 2 x_{3} &{}+{}& x_{4} &{}={}&5\\ 
- x_{1} &{}-{}& x_{2} &{}-{}& 3 x_{3} &{}-{}& 3 x_{4} &{}={}&21\\ 
2 x_{1} &{}-{}& 3 x_{2} &{}-{}& 2 x_{3} &{}-{}& 3 x_{4} &{}={}&2\\ 
2 x_{1} &{}+{}& 2 x_{2} &{}-{}& 2 x_{3} &{}-{}& 2 x_{4} &{}={}&14\\ 
- 2 x_{1} &{}-{}& 2 x_{2} &{}-{}& x_{3} &{}-{}& x_{4} &{}={}&7
\end{alignat*}
\end{enumerate}

\clearpage

\textbf{Exercise 2.} Determine the geometric nature of the subset of $\R^3$ represented by each of the following set of equations. Show work justifying your solution, including details of the solution of the linear systems.

\begin{enumerate}
\item 
\begin{alignat*}{6}
x  &{}-{}& 2y &{}+{}&  z &{}={}&  4\\
2x &{}{}&     &{}-{}& 3z &{}={}& -1\\
   &{}{}&  4y &{}-{}& 5z &{}={}&  9
\end{alignat*}

\item 
\begin{alignat*}{6}
-3x &{}-{}& 3y &{}-{}& 2z &{}={}& -3\\
-8x &{}-{}& 5y &{}-{}& 2z &{}={}&  4\\
 5x &{}+{}& 3y &{}+{}&  z &{}={}& -8
\end{alignat*}

\item 
\begin{alignat*}{6}
 3x &{}{}&     &{}-{}&  z &{}={}&  2\\
 2x &{}-{}& 4y &{}{}&     &{}={}&  5\\
 5x &{}+{}& 8y &{}-{}& 3z &{}={}&  6
\end{alignat*}

\end{enumerate}

\textbf{Exercise 3.} Find, if possible, the inverse of each of the following matrices. To receive full credit, your solution must include the following:
\begin{itemize}
\item The given matrix, augmented by the identity of the same order.
\item The sequence of row operations used to obtain the RREF of the matrix. Write the row operations in the order they were used and the final matrix. It is not necessary to write the intermediate matrices.
\item If the matrix is invertible, your answer must consist of the inverse of the given matrix (which is a square matrix). If the matrix is not invertible, explain how you came to this conclusion.
\end{itemize}

\begin{enumerate}
\item 
\[
\left[\begin{matrix*}[r]-31 & -16 & 4 & 2\\4 & 3 & 0 & 1\\13 & 4 & -3 & -4\\-5 & -2 & 1 & 1\end{matrix*}\right]
\]

\item
\[
\left[\begin{matrix*}[r]17 & \frac{5}{2} & 1\\-4 & - \frac{1}{2} & -1\\7 & 1 & 1\end{matrix*}\right]
\]

\item
\[
\left[\begin{matrix*}[r]1 & 1 & -1 & 1 & 1\\7 & 16 & -5 & 13 & 9\\-9 & -17 & 6 & -13 & -9\\11 & 21 & -7 & 16 & 11\\5 & 9 & -3 & 7 & 5\end{matrix*}\right]
\]
\end{enumerate}

\textbf{Exercise 4.} Find a value of $k$ for which the system below is consistent. You must show all work, including the row operations you used to obtain the RREF of the augmented matrix.
\[
\left[\begin{matrix*}[r]1 & 2 & -3 & 0 \\2 & 2 & -1 & 2 \\3 & -1 & 2 & -2 \\0 & 5 & -6 & 4 & \end{matrix*}\right]
\begin{bmatrix}x\\y\\z\\t\end{bmatrix}=
\begin{bmatrix*}[r] k\\5\\3\\4\end{bmatrix*}
\]

\textbf{Exercise 5.} Determine if each of the following is a linear transformation. Your answer must be justified in terms of the definition of linear transformation.

\begin{enumerate}
\item
\[
L_1\;:\;\R^3\to\R^2
\]
\[
L_1\left(\begin{bmatrix}x\\y\\z\end{bmatrix}\right)=\begin{bmatrix}2x-y+z\\y-z\end{bmatrix}
\]

\item
\[
L_2\;:\;\R^3\to\R^3
\]
\[
L_2\left(\begin{bmatrix}x\\y\\z\\t\end{bmatrix}\right)=\begin{bmatrix}xyt\\x+y-2z\\z\end{bmatrix}
\]

\item
\[
L_3\;:\;\R^2\to\R^4
\]
\[
L_3\left(\begin{bmatrix}x\\y\end{bmatrix}\right)=\begin{bmatrix}x+2y\\y+2\\x-y\\0\end{bmatrix}
\]
\end{enumerate}

\clearpage

\textbf{Exercise 6.} The values of a mystery linear operator $L\;:\;\R^3\to\R^2$ are known for the vectors indicated below:
\[
L\left(\begin{bmatrix*}[r] 1\\ -1\\ 1\end{bmatrix*}\right)=\begin{bmatrix*}[r] 2\\ 0\end{bmatrix*},\quad
L\left(\begin{bmatrix*}[r] -1\\ 0\\-1\end{bmatrix*}\right)=\begin{bmatrix*}[r] -3\\ -1\end{bmatrix*},\quad
L\left(\begin{bmatrix*}[r] 3\\ -2\\ 2\end{bmatrix*}\right)=\begin{bmatrix*}[r] 0\\ 0\end{bmatrix*}.
\]

\textbf{Part (a).} Find how to represent an arbitrary vector of $\R^3$ as:
\[
\begin{bmatrix}x\\y\\c\end{bmatrix}=
c_1\begin{bmatrix*}[r] 1\\ -1\\ 1\end{bmatrix*}+
c_2\begin{bmatrix*}[r] -1\\ 0\\-1\end{bmatrix*}+
c_3\begin{bmatrix*}[r] 3\\ -2\\ 2\end{bmatrix*}
\]
Your answer must show how to find $c_1$, $c_2$ and $c_3$ in terms of $x$, $y$, $z$. To receive full credit, you must justify your solution method, and specify all the steps of your computations.

\textbf{Part (b).} Use the solution to Part (a) and the fact that $L$ is a linear transformation to compute:
\[
L\left(\begin{bmatrix*}[r] 3 \\ 11 \\ -5\end{bmatrix*}\right)=? 
\]

\textbf{Part (c).} Find the value of $L$ for a generic vector in $\R^3$:
\[
L\left(\begin{bmatrix*}[r] x \\ y \\ z\end{bmatrix*}\right)=?
\]
Write your solution in terms of a matrix multiplication.

\textbf{Exercise 7.} Answer the following items based on the following vectors in $\R^3$:

\[
\mathbf{u}=\begin{bmatrix*}[r] 2\\ -3\\ 0\end{bmatrix*},\quad
\mathbf{v}=\begin{bmatrix*}[r] 2\\ 2\\ -3\end{bmatrix*},\quad
\mathbf{w}=\begin{bmatrix*}[r] 1\\ 0\\  2\end{bmatrix*}.
\]

\begin{enumerate} 

\item Compute the vector $2\mathbf{u}-\mathbf{v}-3\mathbf{w}$.

\item Find $||\mathbf{u}+\mathbf{w}||$.

\item Find the cosine of the angle between $\mathbf{u}$ and $\mathbf{v}$.

\item Find, if possible, scalars $a$ and $b$ such that $\mathbf{w}=a\mathbf{u}+b\mathbf{v}$. If it is not possible to find such scalars, explain why.

\item Find a vector:
\[
\begin{bmatrix}x\\y\\z\end{bmatrix}
\]
that is orthogonal to both $\mathbf{u}$ and $\mathbf{v}$. How many such vectors exist?
\end{enumerate}

\textbf{Exercise 8.} Let $\mathbf{u}$ and $\mathbf{v}$ be two arbitrary vectors of $\R^3$:
\[
\mathbf{u}=\begin{bmatrix} u_1\\u_2\\u_3 \end{bmatrix}, \quad
\mathbf{v}=\begin{bmatrix} v_1\\v_2\\v_3 \end{bmatrix}.
\]
The \emph{cross product} of $\mathbf{u}$ and $\mathbf{v}$ is defined by:
\[
\mathbf{u}\times\mathbf{v} =
\begin{bmatrix}u_2v_3-u_3v_2 \\ u_3v_1-u_1v_3\\ u_1v_2-u_2v_1\end{bmatrix}
\]
Show that $\mathbf{u}\times\mathbf{v}$ is orthogonal to both $u$ and $v$.

\textbf{Exercise 9.} In the following items, determine if the given set of vectors is a basis of the corresponding vector space. If the set is a basis, show how to represent an arbitrary vector in terms of the basis. If the set is not a basis, explain why.

\begin{enumerate} 

\item In $\R^4$:
\[
\mathbf{v}_1=\begin{bmatrix*}[r] 1\\ 3\\ -8\\ 5\end{bmatrix*},\quad
\mathbf{v}_2=\begin{bmatrix*}[r] 1\\ 2\\ -3\\ 2\end{bmatrix*},\quad
\mathbf{v}_3=\begin{bmatrix*}[r] -8\\ -3\\ 2\\ -1\end{bmatrix*},\quad
\mathbf{v}_4=\begin{bmatrix*}[r] 5\\ 2\\ -1\\ 1\end{bmatrix*}.
\]

\item In $\R^3$:
\[
\mathbf{u}_1=\begin{bmatrix*}[r] 7\\ 1\\ 2\end{bmatrix*},\quad
\mathbf{u}_2=\begin{bmatrix*}[r] -11\\ -1\\ -3\end{bmatrix*},\quad
\mathbf{u}_3=\begin{bmatrix*}[r] -4\\ 0\\ -1\end{bmatrix*},\quad
\]

\end{enumerate}

\textbf{Exercise 10.} Find the matrix representation of each of the linear transformations below, using the standard basis as both the input and output basis.

\begin{enumerate}

\item 
\[
L_1\;:\;\R^2\to\R^2
\]
\[
L_1\left(\begin{bmatrix} a\\b \end{bmatrix}\right) = \begin{bmatrix} 2a-3b \\ b\end{bmatrix}
\]

\item 
\[
L_2\;:\;\R^3\to\R^4
\]
\[
L_2\left(\begin{bmatrix} x\\y\\z\end{bmatrix}\right) = \begin{bmatrix} z\\x+2y\\0\\x+y+3z\end{bmatrix}
\]

\item 
\[
L_3\;:\;\R^4\to\R^2
\]
\[
L_3\left(\begin{bmatrix} r\\s\\t\\u \end{bmatrix}\right) = \begin{bmatrix} 2r-3s+4t-5v\\ 2t-4s\end{bmatrix}
\]

\end{enumerate}

\textbf{Exercise 11.} The set  $B=\{\mathbf{v}_1,\mathbf{v}_2,\mathbf{v}_3\}$, where:
\[
\mathbf{v}_1=\begin{bmatrix*}[r] 1\\ 0\\ 1\end{bmatrix*},\quad
\mathbf{v}_2=\begin{bmatrix*}[r] 1\\ 1\\ 1\end{bmatrix*},\quad
\mathbf{v}_3=\begin{bmatrix*}[r] 0\\ 1\\ 1\end{bmatrix*}
\]
is a basis of $\R^3$ (this is given, you don't need to verify it). Denote by $E$ the standard basis of $\R^3$. Based on this information, answer the following items:

\begin{enumerate}
\item Find a matrix $P$ that, given any vector representation in base $B$, finds the corresponding representation in base $E$, that is, a matrix such that:
\[
\begin{bmatrix}x\\y\\z\end{bmatrix}_E = P \begin{bmatrix}a\\b\\c\end{bmatrix}_B
\]

\item Find a matrix $Q$ that, given any vector representation in base $E$, finds the corresponding representation in base $B$, that is, a matrix such that:
\[
\begin{bmatrix}a\\b\\c\end{bmatrix}_B = Q \begin{bmatrix}x\\y\\z\end{bmatrix}_E 
\]

\item A linear transformation $L\;:\;\R^3\to\R^3$ is given by:
\[
L\left(\begin{bmatrix}x\\y\\z\end{bmatrix}\right)=\begin{bmatrix}x+y\\y-2z\\3x-y+z\end{bmatrix}
\]
Find the matrix representation of $L$ from the input basis $B$ to the output basis $E$. Recall that this is a matrix $A$ such that, given any vector $\mathbf{u}$,
we can find $L(\mathbf{u})$ by matrix multiplication:
\[
[L(\mathbf{u})]_E=A[\mathbf{u}]_B
\]

\item Find the matrix representation of the linear transformation $L$ from the previous item from the input basis $B$ to the input basis $B$.

\emph{Hint}: This can be solved very quickly if you put together items (2) and (3).
\end{enumerate}

\textbf{Exercise 12.} Let $A$ be the matrix:
\[
\begin{bmatrix} 1 & a & b \\ 0 & 1 & c \\ 0 & 0 & 1\end{bmatrix},
\] 
where $a$, $b$ and $c$ are arbitrary real numbers. Show that this matrix is invertible, and find $A^{-1}$. Your answer will be a matrix in terms of $a$, $b$ and $c$. Provide a full justification with your solution.

\end{document}

























