\documentclass[12pt]{article}

\input{../../../../fimacros.tex}

\setheadings{MTH 288 --- Linear Algebra --- Matrix Powers}

\begin{document}

In this handout we explore a first application of eigenvalues and eigenvectors: computing matrix powers. To give a concrete example, let's consider a simple (and somewhat artificial) population model. We observe the population at discrete times $n=0,1,2,\ldots$, and denote the population at time $n$ by $P_n$. We assume that $P_n$ satisfies the following relations:
\begin{align*}
P_0&=1\\
P_1&=1\\
P_{n+1}&=P_{n}+P_{n-1}\text{ for $n\ge1$}
\end{align*}
The last equation above is called a \emph{recursion}, and we can use it to predict the population at any future time, as illustrated below:
\begin{align*}
P_2&=P_1+P_0=2\\
P_3&=P_2+P_1=3\\
P_4&=P_3+P_2=5\\
P_5&=P_4+P_3=8\\
P_6&=P_5+P_4=13\\
P_7&=P_6+P_5=21\\
\vdots&
\end{align*}
The question we ask is the following: how can we find the size of the population $P_n$ for a given large  value  of $n$ without having to compute all previous values? We will see that we can use eigenvalues and eigenvectors to find the answer.

We introduce the following vector representation for the state of the system:
\[
\mathbf{x}_n=
\begin{bmatrix}P_{n}\\P_{n-1}\end{bmatrix}
\]
We call $\mathbf{x}_n$ the \emph{state vector} of the model. We can now try to find a recursion for the state vector:
\[
\mathbf{x}_{n+1}=
\begin{bmatrix}P_{n+1}\\P_{n}\end{bmatrix}=
\begin{bmatrix}P_{n}+P_{n-1}\\P_{n}\end{bmatrix}=
\begin{bmatrix}1\times P_{n}+1\times P_{n-1}\\1\times P_{n}+0\times P_{n-1}\end{bmatrix}=
\begin{bmatrix}1&1\\1&0\end{bmatrix} \begin{bmatrix}P_{n}\\P_{n-1}\end{bmatrix}=
\begin{bmatrix}1&1\\1&0\end{bmatrix} \mathbf{x}_n.
\]
We define the matrix $A$ by:
\[
A=\begin{bmatrix}1&1\\1&0\end{bmatrix},
\]
and problem can then be translated into the following matrix formulation:
\[
\mathbf{x}_{n+1}=A\mathbf{x}_n
\]
We also introduce the following \emph{initial condition}:
\[
\mathbf{x}_0=\begin{bmatrix}0\\1\end{bmatrix}
\]
Notice that, with this choice, we have:
\[
\mathbf{x}_1=A\mathbf{x_0}=\begin{bmatrix}1&1\\1&0\end{bmatrix}\begin{bmatrix}0\\1\end{bmatrix}=
\begin{bmatrix}0\\1\end{bmatrix}\begin{bmatrix}P_1\\P_0\end{bmatrix},
\]
which is the correct value for $\mathbf{x_1}$.

It is now very easy to write a formula for the vector $\mathbf{x}_n$:
\begin{align*}
\mathbf{x}_{1}&=A\mathbf{x}_0\\
\mathbf{x}_{2}&=A\mathbf{x}_1=A\cdot A\mathbf{x}_0=A^2\mathbf{x}_0\\
\mathbf{x}_{3}&=A\mathbf{x}_2=A\cdot A^2\mathbf{x}_0=A^3\mathbf{x}_0\\
\mathbf{x}_{4}&=A\mathbf{x}_3=A\cdot A^3\mathbf{x}_0=A^4\mathbf{x}_0\\
\vdots&
\end{align*}
and it is easy to see that the general formula is:
\[
\mathbf{x}_{n}=A^n\mathbf{x}_0.
\]
Thus, if we have an efficient way to compute the matrix power $A^n$, we will have a way to find the population for any future time. This can be achieved by diagonalizing the matrix $A$. To understand why, recall that the diagonalization procedure yields a matrix $P$ such that:
\[
D = P^{-1}AP
\]
is a diagonal matrix. Now notice that:
\begin{align*}
D^2&=P^{-1}APP^{-1}AP=P^{-1}A^2P\\
D^3&=DD^2=P^{-1}APP^{-1}APP^{-1}AP=P^{-1}A^3P
\end{align*}
In general:
\[
D^n=P^{-1}A^nP
\]
from which we get:
\[
A^n=PD^np^{-1}
\]
The key point is that powers of a diagonal matrix are very easy to compute. If
$$
D=\begin{bmatrix}a&0\\0&b\end{bmatrix}
$$
then
$$
D^n=\begin{bmatrix}a^n&0\\0&b^n\end{bmatrix}
$$

Going back to the problem of finding an expression for $\mathbf{x}_n$, let's diagonalize the matrix $A$.
We start by finding the eigenvalues of $A$. The characteristic equation is:
\[
\det(A-\lambda I)=\det\begin{bmatrix}1-\lambda&1\\1&-\lambda\end{bmatrix}=(1-\lambda)(-\lambda)-1=\lambda^2-\lambda-1=0,
\]
which has solutions
\[
\lambda=\frac{1\pm\sqrt{1-4(1)(-1)}}{2}=\frac{1\pm\sqrt{5}}{2},
\]
and we get two eigenvalues:
\[
\lambda_1=\frac{1+\sqrt{5}}{2}\text{ and } \lambda_2=\frac{1-\sqrt{5}}{2}.
\]

Notice that $\lambda_1$ is the famous \emph{Golden Ratio}, and $\lambda_2$ is its inverse. This fact no doubt appealed Fibonacci.

We now need to find one eigenvalue for each eigenvector. For each eigenvalue $\lambda$ we have to solve the system:
\[
\begin{bmatrix}1-\lambda & 1\\1 & -\lambda\end{bmatrix}
\begin{bmatrix}x\\y\end{bmatrix}=\begin{bmatrix}0\\0\end{bmatrix},
\]
which gives the system:
\begin{align*}
(1-\lambda)x+y&=0\\
x-\lambda y=0
\end{align*}
We know that this system will have infinitely many solutions, so we can discard one of the equations. Since the second equation is simpler, we use it, which is equivalent to:
\[
x=\lambda y.
\]
Since we can choose any nonzero solution, we let $y=1$ and $x=\lambda$, and get the eigenvector:
\[
\begin{bmatrix}\lambda\\1\end{bmatrix}
\]
Applying this to the eigenvectors $\lambda_1$ and $\lambda_2$ we get:
\begin{align*}
\lambda_1&=(1+\sqrt{5})/2\quad\text{Eigenvector: }\mathbf{v}_1=\begin{bmatrix}(1+\sqrt{5})/2\\1\end{bmatrix}\\
\lambda_2&=(1-\sqrt{5})/2\quad\text{Eigenvector: }\mathbf{v}_2=\begin{bmatrix}(1-\sqrt{5})/2\\1\end{bmatrix}
\end{align*}
Given this information, we construct the matrix that diagonalizes $A$:
\[
P=\begin{bmatrix}(1+\sqrt{5})/2&(1-\sqrt{5})/2\\1&1\end{bmatrix},\quad
P^{-1}=\frac{1}{\sqrt{5}}
\begin{bmatrix}1&-(1-\sqrt{5})/2\\-1&(1+\sqrt{5})/2\end{bmatrix}
\]
Then:
\[
P^{-1}AP=\begin{bmatrix}(1+\sqrt{5})/2&0\\0&(1-\sqrt{5})/2\end{bmatrix}
\]
Putting all together, we get the following formula for the powers of $A$:
\[
A^n=
\frac{1}{\sqrt{5}}\begin{bmatrix}(1+\sqrt{5})/2&(1-\sqrt{5})/2\\1&1\end{bmatrix}
\begin{bmatrix}[(1+\sqrt{5})/2]^n&0\\0&[(1-\sqrt{5})/2]^n\end{bmatrix}
\begin{bmatrix}1&-(1-\sqrt{5})/2\\-1&(1+\sqrt{5})/2\end{bmatrix}
\]

\end{document}











































