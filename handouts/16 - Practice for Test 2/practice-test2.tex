\documentclass[12pt]{article}

\input{../../../../fimacros.tex}

\setheadings{MTH288 --- Practice for Test 2}

\begin{document}

\bigskip
\textbf{Exercise 1.} Compute the determinant of the following matrices. Do not use the $\mathtt{det()}$ method. Indicate the method you used to compute the determinant, and show all computations:

\medskip
(a) $\left[\begin{matrix*}[r]1 & 2\\-3 & 1\end{matrix*}\right]$

\medskip
(b) $\left[\begin{matrix*}[r]2 & 0 & -3\\1 & 2 & 0\\3 & -1 & 2\end{matrix*}\right]$

\bigskip
\textbf{Exercise 2.} $A$ is a $5\times 5$ matrix and it is known that $\det(A)=-2$. The matrix $B$ is obtained by applying the following operations to $A$:
\begin{itemize}
\item Multiply row $3$ row by $2$.
\item Add to column $4$ the result of multiplying column $1$ by $-2$.
\item Swap columns $2$ and $5$.
\item Add row $3$ to row $4$.
\item Multiply column $5$ by $-3$.
\item Swap rows $1$ and $3$. 
\end{itemize} 
Find $\det(B)$, and justify your answer.

\bigskip
\textbf{Exercise 3.} Let:
\[
\mathbf{v}_1=\left[\begin{matrix*}[r]1\\-2\\2\\3\end{matrix*}\right],\quad
\mathbf{v}_2=\left[\begin{matrix*}[r]0\\1\\0\\-1\end{matrix*}\right],\quad
\mathbf{v}_3=\left[\begin{matrix*}[r]2\\-7\\4\\9\end{matrix*}\right],\quad
\mathbf{v}_4=\left[\begin{matrix*}[r]1\\-1\\0\\3\end{matrix*}\right]
\]
Let
\[
V=\text{span}\{\mathbf{v}_1,\mathbf{v}_2,\mathbf{v}_3,\mathbf{v}_4\}
\]

\medskip
(a) Find a basis of $V$. What is the dimension of $V$?

\medskip
(b) Determine if the vector $\mathbf{w}=\left[\begin{matrix}3\\-9\\4\\13\end{matrix}\right]$ is in $V$. If it is, write it as a linear combination of the vectors in the basis you found in the previous item.

\medskip
(c) Find a vector in $\mathbb{R}^4$ that is not in $V$.

\bigskip
\textbf{Exercise 4.} Let
\[
\mathbf{v}_1=\left[\begin{matrix*}[r]0\\-1\\3\\-2\end{matrix*}\right],\quad
\mathbf{v}_2=\left[\begin{matrix*}[r]2\\0\\1\\3\end{matrix*}\right],\quad
\mathbf{v}_3=\left[\begin{matrix*}[r]4\\-1\\2\\1\end{matrix*}\right],\quad
\mathbf{v}_4=\left[\begin{matrix*}[r]2\\0\\1\\0\end{matrix*}\right]
\]

\medskip
(a) Show that $B=\{\mathbf{v}_1,\mathbf{v}_2,\mathbf{v}_3,\mathbf{v}_4\}$ is a basis of $\mathbb{R}^4$.

\medskip
(b) Find the change of basis matrix from basis $B$ to basis $E$, the standard basis of $\mathbb{R}^4$. 

\medskip
(c) Find the change of basis matrix from basis $E$ to basis $B$.

\medskip
(d) Find the coordinates in basis $B$ of the vector
$\begin{bmatrix*}[r]1\\-3\\2\\0\end{bmatrix*}_E$. 

\bigskip
\textbf{Exercise 5.} For each of the matrices below, find all eigenvalues and a basis for each eigenspace. Then, determine if the matrix is diagonalizable. If it is, find a matrix $P$ such that $D=P^{-1}AP$ is diagonal, and compute $P^{-1}AP$ to verify that your solution is correct.

\medskip
(a) $A=\left[\begin{matrix*}[r]2 & 0 & 1\\-3 & 5 & -3\\-6 & 6 & -5\end{matrix*}\right]$

\medskip
(b) $A=\left[\begin{matrix*}[r]0 & 5 & -2\\-3 & -8 & 2\\-5 & -9 & 1\end{matrix*}\right]$

\medskip
(c) $A=\left[\begin{matrix}13 & -12 & 21\\15 & -14 & 21\\0 & 0 & -2\end{matrix}\right]$

\medskip
(d) $A=\left[\begin{matrix*}[r]-10 & 0 & -10 & 8 & -19\\3 & 1 & 26 & -8 & 11\\-1 & 0 & 3 & 0 & -1\\0 & 0 & 16 & -3 & 4\\5 & 0 & 6 & -4 & 10\end{matrix*}\right]$

\medskip
(e) $A=\left[\begin{matrix}-16 & -72 & 27 & 3\\0 & -10 & 9 & -9\\-9 & -66 & 38 & -21\\-9 & -54 & 27 & -10\end{matrix}\right]$

\bigskip
\textbf{Exercise 6.} Let
\[
\mathbf{v}_1=\begin{bmatrix*}[r]2\\-1\\0\\3\end{bmatrix*},\quad
\mathbf{v}_2=\begin{bmatrix*}[r]1\\2\\-2\\0\end{bmatrix*}.
\]
Find a basis for the subspace of all vectors $\mathbf{u}=\begin{bmatrix*}[r]x\\y\\z\\t\end{bmatrix*}$ in $\R^4$ that are orthogonal to both $\mathbf{v}_1$ and $\mathbf{v}_2$

\bigskip
\textbf{Exercise 7.} Suppose that $\mathbf{u}$ and $\mathbf{v}$ are two vectors in $\R^n$ such that:
\begin{itemize}
\item $\mathbf{u}$ and $\mathbf{v}$ are orthogonal.
\item $\mathbf{u}+\mathbf{v}$ and $\mathbf{u}-\mathbf{v}$ are also orthogonal.
\end{itemize}
Given this information, what can you conclude about $||\mathbf{u}||$ and $||\mathbf{v}||$? Justify your answer.

\bigskip
\textbf{Exercise 8.} In each of the items below, use the Gram-Schmidt process to find an orthonormal basis of the subspace spanned by the given vectors.

\medskip
(a)
$
\mathbf{v}_1=\begin{bmatrix*}[r]1\\-2\\0\end{bmatrix*}, 
\mathbf{v}_2=\begin{bmatrix*}[r]0\\2\\-4\end{bmatrix*} 
$

\medskip
(b)
$
\mathbf{v}_1=\begin{bmatrix*}[r]0\\0\\1\end{bmatrix*}, 
\mathbf{v}_2=\begin{bmatrix*}[r]2\\-2\\1\end{bmatrix*}, 
\mathbf{v}_3=\begin{bmatrix*}[r]1\\0\\3\end{bmatrix*}
$


\medskip
(c)
$
\mathbf{v}_1=\begin{bmatrix*}[r]2\\-1\\0\\0\end{bmatrix*}, 
\mathbf{v}_2=\begin{bmatrix*}[r]3\\-3\\1\\1\end{bmatrix*}, 
\mathbf{v}_3=\begin{bmatrix*}[r]0\\2\\0\\1\end{bmatrix*}
$


\bigskip
\textbf{Exercise 9.} For each of the following items, do the following:
\begin{itemize}
\item Find an orthonormal basis consisting of eigenvectors of the symmetric matrix $A$.
\item Find a matrix $P$ such that $D=P^TAP$ is a diagonal matrix.
\item Compute the product $P^TAP$ to confirm that it is equal to a diagonal matrix with the eigenvalues of $A$ on its diagonal.
\end{itemize}

\medskip
(a)
$A=\left[\begin{matrix*}[r]0 & -1 & -1\\-1 & 1 & -2\\-1 & -2 & 1\end{matrix*}\right]$

\medskip
(b)
$A=\left[\begin{matrix*}[r]-2 & -5 & 5\\-5 & -2 & 5\\5 & 5 & -2\end{matrix*}\right]$

\medskip
(c)
$A=\left[\begin{matrix*}[r]-4 & 3 & -2 & -5\\3 & -4 & -2 & 5\\-2 & -2 & 1 & 0\\-5 & 5 & 0 & -2\end{matrix*}\right]$

%\bigskip
%\textbf{Exercise 10.} 

%\bigskip
%\textbf{Exercise 11.} 

%\bigskip
%\textbf{Exercise 12.} 

\end{document}

























