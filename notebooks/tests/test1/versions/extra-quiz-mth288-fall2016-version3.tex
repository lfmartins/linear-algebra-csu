\documentclass[12pt]{article}

\input{../../../../../fimacros.tex}

\setheadings{MTH288 --- Linear Algebra --- Test 1}


\begin{document}

Name: \hrulefill

\bigskip
Student ID: \hrulefill


\textbf{Problem 1.} (10 points)
The values of a ``mystery'' linear transformation $L:\R^3\to\R^2$ are known for three vectors:
\[
L\left(\left[\begin{matrix}-1\\-1\\1\end{matrix}\right]\right)=\left[\begin{matrix}0\\-1\end{matrix}\right],\quad 
L\left(\left[\begin{matrix}3\\0\\1\end{matrix}\right]\right)=\left[\begin{matrix}3\\3\end{matrix}\right],\quad 
L\left(\left[\begin{matrix}3\\1\\0\end{matrix}\right]\right)=\left[\begin{matrix}0\\-1\end{matrix}\right]
\]

\textbf{Part (a).} Given an arbitrary vector, find scalars $c_1$, $c_2$ and $c_3$ such that:
\[
\begin{bmatrix}x\\y\\z\end{bmatrix}=c_1\left[\begin{matrix}-1\\-1\\1\end{matrix}\right]+c_2\left[\begin{matrix}3\\0\\1\end{matrix}\right]+c_3\left[\begin{matrix}3\\1\\0\end{matrix}\right]
\]
Your answer should give the values of $c_1$, $c_2$ and $c_3$ as a function of $x$, $y$ and $z$. Notice that the transformation $L$ is not used to solve this part of the problem.

\vfill
(This problem continues on the next page.)

\clearpage

\textbf{Part (b).} Use the previous item to evaluate:
\[
L\left(\left[\begin{matrix}-7\\-3\\1\end{matrix}\right]\right)
\]
To receive full credit, your solution must show how to use the answer to the previous item and the linearity property of $L$ to obtain the result. 


\end{document}
























