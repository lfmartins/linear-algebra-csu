\documentclass[12pt]{article}

\input{../../../../../fimacros.tex}

\setheadings{MTH288 --- Linear Algebra --- Test 1}


\begin{document}

Name: \hrulefill

\bigskip
Student ID: \hrulefill

\bigskip
\textbf{Instructions.} All solutions must be justified, unless otherwise stated. Show all work leading to your answer in each problem. Solutions without appropriate work that supports it will receive no credit. All work must be written in the test. Do not attach computer printouts to the test. If not enough space is provided for an answer, continue it in the back of the page. All answers must be given as exact values, decimal approximations will not be accepted.

Please identify your final answer to each problem by surrounding it with a rectangle.

\vfill
\hfill$\spadesuit$

\clearpage

% Problem 1
\textbf{Problem 1.} (30 points)
Answer the following items for the system below. Please read carefully what is being asked in each item!
\begin{alignat*}{13}
- x_{1} &{}-{}& 3 x_{2} &{}+{}& 13 x_{3} &{}-{}& x_{4} &{}-{}& 3 x_{5} &{}={}&9\\ 
- 3 x_{1} &{}-{}& 3 x_{2} &{}+{}& 15 x_{3} &{}+{}& 2 x_{4} &{}-{}& 3 x_{5} &{}={}&2\\ 
- x_{1} &{}-{}& 2 x_{2} &{}+{}& 9 x_{3} &{}+{}& x_{4} &{}+{}& 2 x_{5} &{}={}&-5\\ 
x_{1} &{}-{}& 2 x_{2} &{}+{}& 7 x_{3} &{}-{}& x_{4} &{}-{}& 3 x_{5} &{}={}&2\\ 
\end{alignat*}

\textbf{Part (a).} (5 points) Write in the space below the augmented matrix corresponding to this system.

\vskip1.5in

\textbf{Part (b).} (10 points) Use elementary row operations to find the RREF of the augmented matrix. In the space below, enter both the sequence of row operations used and the resulting matrix in RREF.

\clearpage

\textbf{Part (c).} (5 points) Write the system of equations that correspond to the RREF matrix from the previous item.

\vskip3in

\textbf{Part (d).} (10 points) Write the solution set of the system. Use a parametric representation, with one parameter for each free variable.

\clearpage


% Problem 2
\textbf{Problem 2.} (25 points) Answer the following items for the square matrix $A$ given below. Please read carefully what is being asked in each item!
\[
\left[\begin{matrix}-1 & -1 & -1\\3 & -2 & -1\\3 & -1 & 0\end{matrix}\right]
\]
\textbf{Part (a).} (5 points) Write the augmented matrix used to compute the inverse of the matrix A.

\vskip1in

\textbf{Part (b).} (15 points) Use elementary row operations to find the RREF of the augmented matrix.
In the space below, write both the sequence of row operations and the resulting matrix in RREF.

\vskip3.5in

\textbf{Part (c).} (5 points) In the space below, enter the inverse of the matrix $A$.

\clearpage

% Problem 3
\textbf{Problem 3.} (10 points) Answer the following items for the vectors below:
\[
\mathbf{v}_1=\left[\begin{matrix}1\\3\\1\end{matrix}\right],\quad
\mathbf{v}_2=\left[\begin{matrix}-2\\1\\-1\end{matrix}\right]
\]
\textbf{Part (a).} (5 points) Compute $||2\textbf{v}_1-\textbf{v}_2||$.

\vskip3.5in

\textbf{Part (b).} (5 points) Determine if $\textbf{v}_1$ and $\textbf{v}_2$ are orthogonal.

\clearpage

% Problem 4
\textbf{Problem 4.} (25 points) The values of a ``mystery'' linear transformation $L:\R^2\to\R^2$ are known for two vectors:
\[
L\left(\left[\begin{matrix}4\\-3\end{matrix}\right]\right)=\left[\begin{matrix}-3\\4\end{matrix}\right],\quad L\left(\left[\begin{matrix}7\\-5\end{matrix}\right]\right)=\left[\begin{matrix}7\\6\end{matrix}\right]
\]

\textbf{Part (a).} (15 points) Given an arbitrary vector $\begin{bmatrix}x\\y\end{bmatrix}$, find scalars $a$ and $b$ such that:
\[
\begin{bmatrix}x\\y\end{bmatrix}=a\left[\begin{matrix}4\\-3\end{matrix}\right]+b\left[\begin{matrix}7\\-5\end{matrix}\right]
\]
Your answer should give the values of $a$ and $b$ as a function of $x$ and $y$.

\vfill
(This problem continues on the next page.)

\clearpage

\textbf{Part (b).} (10 points) Use the previous item to evaluate:
\[
L\left(\begin{bmatrix}5\\-2\end{bmatrix}\right)
\]
To receive full credit, your solution must show how to use the answer to the previous item and the linearity property of $L$ to obtain the result. 

\clearpage

% Problem 5
\textbf{Problem 5.} (10 points) A function $L:\R^2\to\R$ is defined by:
\[
L\left(\begin{bmatrix}x\\y\end{bmatrix}\right)=-2x+xy
\]
Determine if $L$ is a linear transformation, and justify your answer. To receive full credit, your justification must use the definition of linear transformation.

\end{document}
























